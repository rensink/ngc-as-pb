\section{Provable entailment}
\slabel{entailment}

\todo{AR: Section has to be adapted to new structure}
Given the number of possible forward and backward shifters we identified, there is a lot of choice within the concept of a morphism, possibly making them complex to reason about. However, we now set about showing that, in an important sense, it is enough to consider only direct forward-shift morphisms. To make this statement precise, we introduce the notion of \emph{provable entailment}, which are those entailments that can be established through the existence of only (complete) forward-shift morphisms.

\begin{definition}[provable entailment/equivalence]\dlabel{provable}
\emph{Provable entailment}, denoted $\pentails$, is the smallest preorder such that for all sb-conditions $b,c$:
\begin{itemize}[topsep=\smallskipamount]
\item If there is a direct forward-shift morphism $m:b\to c$, then $b\pentails c$;
\item If there is a complete direct forward-shift morphism $m:b\to c$, then $c\pentails b$.
\end{itemize}
\emph{Provable equivalence}, denoted $\pequiv$, is defined as ${\pequiv}={\pentails}\cap{\pentailedby}$.
\end{definition}
%
The above claim now comes down to the fact that any entailment proved by the existence of a forward or backward morphism, complete or not, is already provable in the sense of $\pentails$, hence using direct forward-shift morphisms only. This is formalised in the final main theorem of the paper

\begin{theorem}[provable entailment]\thlabel{pentails}
Let $b,c$ be sb-conditions. If there exists an sb-condition morphism $m:b\to c$, then $b\pentails c$; moreover, if $m$ is complete then $c\pentails b$.
\end{theorem}
%
The rest of this section is devoted to proving this theorem. For this we need a number of auxiliary properties, starting with the following congruence:

\begin{proposition}\plabel{pequiv congruent}
Let $b,c$ be sb-conditions with $|b|=|c|$ and $s^b_i=s^c_i$ for all $1\leq i\leq |b|$.
\begin{enumerate}[topsep=\smallskipamount]
\item If $b_i\pentails c_i$ for all $i=1,\ldots,|b|$ then $c\pentails b$;
\item If $b_i\pequiv c_i$ for all $i=1,\ldots,|b|$ then $b\pequiv c$.
\end{enumerate}
\end{proposition}
%
\begin{proof}\todo{AR: added AC: I'm not convinced... :-( AR: extended}
Each $b_i\pentails c_i$ follows from the transitive closure of a sequence of primitive steps (i.e., one of the two cases of \dcite{provable}). It therefore suffices to prove that, if \emph{a single} $p_i\pentails c_i$ is due to the existence of either
\begin{enumerate*}[label=(\roman*)]
\item a forward [backward] shift morphism $m_i:b_i\to c_i$ or
\item a complete forward [backward] shift morphism $c_i\to b_i$,
\end{enumerate*}
whereas $b_j=c_j$ for all $j\leq i$, then there exists either
\begin{enumerate*}[label=(\roman*)]
\item a forward [backward] shift morphism $m:c\to b$ or 
\item a complete forward [backward] shift morphism $m:b\to c$,
\end{enumerate*}
implying, in both cases, $c\pentails b$.

\smallskip
For both cases of the proof, let $m_j=\id_{b_j}$ for $j\leq i$, and let
%
\[ m = (\id_{[1,w]}((\id_{P^b_1},m_1)\ccdots (\id_{P^b_w},m_w)) \]
%
\begin{enumerate}[label=(\roman*)]
\item In this case, each $m_j$ (for $1\leq j\leq w$) is a forward [backward] morphism from $b_j$ to $c_j$; hence $m$ is a morphism from $c$ to $b$.\todo{AR: This uses the ``fact" that identity morphisms are both complete backward and complete backward shift morphisms. Should we state that explicitly?}

\item In this case, each $m_j$ (for $1\leq j\leq w$) is a complete morphism from $c_j$ to $b_j$; hence $m$ is a complete morphism from $c$ to $c$.
\qed
\end{enumerate}
\end{proof}
%
Next, there is a small but important fact about span source shifters: $\rF v x$ undoes the effect of $\dB v$, and $\dF v$ undoes the effect of $\rB v x$.\todo{AR: This could also be included in the section about span source shifters} The proof is immediate from the construction (see \fcite{sb-shifters}).
%
\begin{lemma}\llabel{sb-shifter inverses}
If $v:A\to B$ is a split epi with section $x$, then $\rF v x;\dB v$ and $\dF v;\rB v x$ are the identity on $B$-sourced spans.
\end{lemma}
%
Next, we observe that a pattern shift along an identity morphism is conservative if the mediating morphism $k$ is epi.

\begin{lemma}\llabel{id-pattern shift}
If the arrow $k$ in the following diagram is epi, then $(\spanof{u_1}{d_1},\spanof{u_2}{d_2})\in \ccP_{\id_P}$.
\begin{center}
\begin{tikzpicture}[on grid,baseline=(I1.center)]
\node (A) {$A$};
\node (I1) [below left=1 and 1.5 of A] {$I_1$};
\node (P1) [below=1.5 of I1] {$P$};
\node (I2) [below right=1 and 1.5 of A] {$I_2$};
\node (P2) [below=1.5 of I2] {$P$};
\path
  (I1) edge[->] node[above left,inner sep=1] {$u_1$} (A)
  (I1) edge[->] node[left] {$d_1$} (P1)
  (I2) edge[->] node[above right,inner sep=1] {$u_2$} (A)
  (I2) edge[->] node[right] {$d_2$} (P2);
\path[morphism]
  (P1) edge[->] node[above] {$\id_P$} (P2);
\path[red,color=red]
  (I1) edge[->>] node[below] {$k$} (I2);
\end{tikzpicture}
\end{center}
\end{lemma}
%
\begin{proof}
This follows from the fact that the lower square in the diagram is a pushout.\todo{A: Expand?}
\end{proof}
%
Using this, we can observe the following interplay of conservative pattern shifting and various source shifters.

\begin{lemma}\llabel{BF}\todo{AR: statement repaired, proof to be done}
Let $v:A\to B$ be an arrow.
\begin{enumerate}[topsep=\smallskipamount]
\item For any forward span source shifter $\cF$ for $v$ and any $A$-sourced span $s$, $(\dF v(s),\cF(s))\in \ccP_\id$.
\item For any backward span source shifter $\cB$ for $v$ and any $B$-sourced span $s$, $(\dF v(\cB(s)),s)\in \ccP_\id$.
\end{enumerate}
\end{lemma}
%
\begin{proof}
We prove both clauses for elementary source shifters (i.e., of the form $\rF v x$, $\dF v$, $\rB v x$ or $\dB v$); the case for arbitrary source shifters then follows from \asscite{sb-shifters} in combination with \pcite{sb-shifters compose} (clause \ref{pattern-transitive}) and \pcite{pattern shift} (clause \ref{conservative-transitive}). Let $v:A\to B$ be an arrow.
\begin{enumerate}[topsep=\smallskipamount]
\item The case of $\cF=\dF v$ is immediate, since $\ccP_\id$ includes the identity over $B$-sourced spans.

Now consider $\cF=\rF v x$ (where $x;v=\id_B$). Let $s=\spanof u d$, $\rF v x=s_1=\spanof{u_1}{d_1}$ and $\dF v(s)=s_2=\spanof{u_2}{d_2}=\spanof{u;v}{d}$. The pair $(\dF v(s),\rF v x(s))$ forms a pair as in \lcite{id-pattern shift}, with $k$ being the pullback of $v$ over $u$.

Now $k=x'$ as in \fcite{sb-source shifters} shows $(s_1,s_2)\in \cP_\id$ according to \eqcite{pattern shift}.

\item First consider $\cB=\rB v x$ where $x;v=\id_B$. From \lcite{sb-shifter inverses} we know that $\dF v(\cB(s))=s$ for all $s$; since $\ccP_\id$ includes the identity over $B$-sourced spans, we are done.

Now consider $\cB=\dB v$. From \lcite{sb-shifter inverses} we know that $\rF v s(\dB v(s))=s$ for all $s$. Given that $(\dB v(s),\dB v(s))\in \ccP_\id$ and that $\rF v x$ preserves $\ccP$, it follows that $(\rF v x(\cB(s)),s)=(\rF v x(\dB v(s)),$


This time $\dF v$ does not act as the inverse of $\cB$. Rather, if $s=\spanof{u_1}{d_1}:B\to C$ with interface object $I_1$ and $\dF v(\dB v(s))=\spanof{u_2}{d_2}$ with interface object $I_2$, then the diagram on the left hand side arises:\todo{AC: in the diagram all 1's and 2's are switched, right? Also in the right part that has to be fixed.}
\begin{center}
\begin{tikzpicture}[on grid]
\node (A) {$A$};
\node (B) [right=2 of A] {$B$};
\node (I1) [below left=of B] {$I_1$};
\node (I2) [below right=of B] {$I_2$};
\node (C) [below right=of I1] {$C$};
\path
  (A) edge[->>] node[above] {$v$} (B)
  (I1) edge[->] node[above left] {$u_1$} (B)
  (I1) edge[->] node[below left] {$d_1$} (C)
  (I2) edge[->] node[above right] {$u_2$} (B)
  (I2) edge[->] node[below right] {$d_2$} (C);
\path[red,color=red]
  (I1) edge[->>] node[above] {$k$} (I2);
\end{tikzpicture}
\qquad
\begin{tikzpicture}[on grid]
\node (I1) {$I_1$};
\node (I2) [right=1.5 of I1] {$I_2$};
\node (C1) [below=of I1] {$C$};
\node (C2) [below=of I2] {$C$};
\path
  (I1) edge[->>] node[above] {$k$} (I2)
  (I1) edge[->] node[left] {$d_1$} (C1)
  (C1) edge[->] node[above] {$\id$} (C2)
  (I1) edge[->] node[right] {$d_1$} (C2);
\end{tikzpicture}
\end{center}
Note that $v$ has to be epi for $\dB v$ to exist, and $k$ is then epi because, in toposes, epis are preserved by pullbacks. The diagram shows $(\dF v(\dB v(s)),s)\in \cP_\id$.

For the pattern shift to be conservative, we need the bottom ``square" (with $\id_C$ as the arrow along the bottom).\todo{AR: not quite done} This square is explicitly drawn on the right hand side. Indeed, the pushout nature follows from the fact that $k$ is epi.\todo{AR: Include the arrows that make up the pushout argument?}
\qed
\end{enumerate}
\end{proof}
%
This sets the stage for the following (technical) result, which in turn directly implies \thcite{pentails}.

\begin{lemma}\llabel{sb-forward only}
For any $m$, let $\bar m$ be derived from $m$ by replacing all branch mappings $o$ (on all levels) by the corresponding identities.

\smallskip
Let $b\in \SC A,c\in \SC B$ be sb-conditions, and let $v:A\to B$ be an arrow.
\begin{enumerate}[topsep=\smallskipamount]
\item For any [complete] forward root shifter $\cF$ for $v$, if $m$ is a [complete] forward-shift morphism from $\cF(b)$ to $c$, then there are sb-conditions $b'\pequiv b$ and $c'\pequiv c$ such that $\bar m$ is a direct [complete] forward-shift morphism from $\dF v(b')$ to $c'$.

\item For any [complete] backward root shifter $\cB$ for $v$, if $m$ is a [complete] backward-shift morphism from $b$ to $\cB(c)$, then there are sb-conditions $b'\pequiv b$ and $c'\pequiv c$ such that $\bar m$ is a direct [complete] forward-shift morphism from $\dF v(b')$ to $c'$.
\end{enumerate}
\end{lemma}
%
\begin{proof}
Let $b\in \SC A,c\in \SC B$ be sb-conditions, and let $v:A\to B$ be an arrow. The proof is by induction on the depth of $m$.
\begin{enumerate}[topsep=\smallskipamount]
\item \todo{To be redone in the same style as the second clause.}

\smallskip
If $m$ is complete, then all its root shifters are complete, and hence already direct.

\item Let $\cB$ be a backward root shifter for $v$, and let $m=(o,(v_1,m_1)\ccdots(v_w,m_w)): b \to \cB(c)$ a backward-shift morphism. It follows that all $m_i$ are morphisms from $c_{o(i)}$ to $\cB_i(b_i)$ for some backward shifter $\cB_i$. By the induction hypothesis therefore (letting $j=o(i)$), $\bar m_i:\dF{v_i}(c_j')\to b_i'$ for some $c_j'\pequiv c_j$ and $b'_i\pequiv b_i$.

\smallskip
In constructing $c'$ from $c$, we have to make sure that we retain the $c$-branches that are not the image of any $b$-branch under $o$. For this purpose, let $J=\setof{1\leq j\leq |c|\mid \nexists i\st o(i)=j}$ be the set of indexes of those branches, and let $j_1,\ldots,j_n$ be an arbitrary ordering of the elements of $J$. We define 
\begin{align*}
b'=\;&(A,(s^b_1,b'_1)\ccdots (s^b_w,b'_w)) \\
c'=\;&(B,p_1\ccdots p_w\,p^c_{j_1}\ccdots p^c_{j_n}) \\
   & \quad\text{where } p_i = (\dF v(\cB(s^c_{o(i)})),c'_{o(i)}) \text{ for all $1\leq i\leq w$}\enspace.
\end{align*}
%
Note that $|c'|=w+n$. For $1\leq i\leq w$, the $i$th branch of $c'$ is based on the $o(i)$-th branch of $c$, modifying its span by first backward-shifting it using $\cB$ and then forward-shifting using $\dF v$, and modifying its subcondition by taking, instead of $c_{o(i)}$, the provably equivalent $c'_{o(i)}$. For $w+1\leq i\leq w+n$, the $i$th branch of $c'$ is the $j_i$-th branch of $c$.

\smallskip
$\bar m$ (which has $o^{\bar m}=\id_{[1,w]}$) is a direct forward-shift morphism from $\dF v(b')$ to $c'$. To show this, all that is yet required is that $(\dF v(s^b_i),\dF v(\cB(s^c_{o(i)}))) \in \cP_{v_i}$ for $1\leq i\leq w$. This follows from the fact that $m:b\to \cB(c)$ is a morphism, which means $(s^b_i,\cB(s^c_{o(i)}))\in \cP_{v_i}$, in combination with \pcite{pattern shift}.\ref{pattern-congruence}, which states that the action of $\dF v$ on spans (which comes down to prefixing them with $\spanof{v}{\id}$, see \fcite{sb-source shifters}) preserves $\cP_{v_i}$.

\smallskip
It only remains to be shown that $b'\pequiv b$ and $c'\pequiv c$. The first directly follows from \lcite{pequiv}, given that $b'_i\pequiv b_i$ for all $1\leq i\leq w$. For the second, we go through the auxiliary conditions
\begin{align*}
c''=\;&(B,(s^c_{o(1)},c'_{o(1)})\ccdots (s^c_{o(w)},c'_{o(w)})\,p^c_{j_1}\ccdots p^c_{j_n}) \\
c'''=\;&(B,(s^c_{o(1)},c_{o(1)})\ccdots (s^c_{o(w)},c_{o(w)})\,p^c_{j_1}\ccdots p^c_{j_n}) \enspace.
\end{align*}
\begin{itemize}
\item $c'\pequiv c''$ because $\id_{c''}$ is a direct complete forward-shift morphism from $c''$ to $c'$ due to $(\dF v(\cB(s^c_{o(i)})),s^c_{o(i)})\in \ccP_\id$.\footnote{Note that this reuses the mathematical object $\id_{c''}$, not in its role as identity morphism $c''\to c''$ but to relate $c''$ to $c'$.}\todo{AR: does the footnote make sense?} \todo{AC: this also looks switched wrt Lemma 4.18. AR: yes, but that just means the morphism goes in the opposite direction} for all $1\leq i\leq w$ (see \lcite{BF});

\item $c''\pequiv c'''$ due to \lcite{pequiv}, given that $c'_{o(w)}\pequiv c_{o(w)}$ for all $1\leq i\leq w$; 

\item $c'''\pequiv c$ because $m'=(o',(\id_{P^c_1},\id_{c_1})\ccdots (\id_{P^c_{|c|}},\id_{c_{|c|}}))$ with $o'=o^m\cup\setof{(w+i,j_i) \mid 1\leq i\leq n}$ is a direct complete forward-shift morphism from $c'''$ to $c$.
\qed
\end{itemize}
If $m$ is complete, then so is $\bar m$, because in that case $n=0$ (there are no $c$-branches that are not the image of any $b$-branch) and so $\id_{[1,w]}$ is surjective in $\bar m:\dF v(b')\to c'$.
\end{enumerate}
\end{proof}

%
%Motivated by \pcite{sb-forward only}, from now on we drop the qualifiers ``forward-shift" and ``backward-shift" and just talk about (sb-)morphisms, always meaning those that are based on direct forward shifting.
