%% Additional packages
\usepackage[inline]{enumitem}
\usepackage{mathdots}
\usepackage{mathtools}
\usepackage{comment}
\usepackage{subcaption}
\usepackage{hyperref}
%\usepackage[svgnames]{xcolor}

%% Some math definitions
\newcommand{\tupof}[1]{\langle{#1}\rangle}
\newcommand{\setof}[1]{\{{#1}\}}
% span <a,b> cospan >a,b<
%\newcommand{\spanof}[2]{\tupof{{#1},{#2}}}
%\newcommand{\cospanof}[2]{\rangle{#1},{#2}\langle}
% span <a|b> cospan >a|b<
%\newcommand{\spanof}[2]{\tupof{{#1}|{#2}}}
%\newcommand{\cospanof}[2]{\rangle{#1}|{#2}\langle}
% span a<>b cospan a><b
%\newcommand{\spanof}[2]{{#1}\lrangle{#2}}
%\newcommand{\cospanof}[2]{{#1}\rlangle{#2}}
% span a<>b cospan a><b, using diamond and times symbols
%\newcommand{\spanof}[2]{{#1}{\mathsmaller\diamond}{#2}}
%\newcommand{\cospanof}[2]{{#1}{\mskip-1mu\mathsmaller\times\mskip-1mu}{#2}}
% span <|a,b|> and cospan |>a,b<|
\newcommand{\spanof}[2]{\lmangle{#1},{#2}\mrangle}
\newcommand{\cospanof}[2]{\mrangle{#1},{#2}\lmangle}

% span composition
%\newcommand{\spcomp}{\bbsemicolon}
\newcommand{\spcomp}{\fatsemi}
% \newcommand{\spcomp}{;}



% skip distance before langle or after rangle
\newcommand{\angleskip}{%
\mathchoice{\mskip-1mu}%
           {\mskip-1mu}%
           {\mskip-1.1mu}%
           {\mskip-1.2mu}%
}
% skip distance between angle and mid symbol
\newcommand{\anglemidskip}{%
\mathchoice{\mskip-3.8mu}%
           {\mskip-3.8mu}%
           {\mskip-4.1mu}%
           {\mskip-4.6mu}%
}
% langle followed by adjacent rangle
\newcommand{\lrangle}{{%
\angleskip
\langle
\mathchoice{\mskip-2.65mu}%
           {\mskip-2.65mu}%
           {\mskip-3.1mu}%
           {\mskip-3.65mu}%
\rangle
\angleskip
}}
% rangle followed by adjacent langle
\newcommand{\rlangle}{{%
\angleskip
\rangle
\mathchoice{\mskip-4.5mu}%
           {\mskip-4.5mu}%
           {\mskip-5mu}%
           {\mskip-5.3mu}%
\langle
\angleskip
}}
% langle followed by adjacent mid
\newcommand{\lmangle}{{\langle\anglemidskip|}}
% mid followed by adjacent rangle
\newcommand{\mrangle}{{|\anglemidskip\rangle}}

\newcommand{\commutes}{\circlearrowleft}

\newcommand{\st}{.\;}
\newcommand{\id}{\mathit{id}}
\newcommand{\depth}{\mathit{dp}}
\newcommand{\fv}{\mathit{fv}}

\newcommand{\sat}{\models}
\newcommand{\nsat}{\not\models}
\newcommand{\entails}{\models}
\newcommand{\nentails}{\not\models}

\newcommand{\pentails}{\models^{\mathsf p}}
\newcommand{\pentailedby}{\Relbar\joinrel\mathrel{|}^{\mathsf p}}
\newcommand{\pequiv}{\equiv^{\mathsf p}}

\newcommand{\degr}[1]{|{#1}|}
\newcommand{\ccdots}{{\cdotp\mkern-2mu\cdotp\mkern-2mu\cdotp\mkern1mu}}

% some more
% renew because \cat define in mypaper.tex referenced by preample.sty 
\renewcommand{\cat}[1]{\textbf{#1}}
%\newcommand{\shift}[2]{(#1{\uparrow} #2)}
%\newcommand{\bshift}[2]{(#1{\downarrow} #2)}

\newcommand{\Rshift}{\mathbin{\nearrow_{\!\!\!\mathsf r}}}
\newcommand{\Rbshift}{\mathbin{\swarrow_{\!\!\!\mathsf r}}}
\newcommand{\Ishift}{\mathbin{\nearrow_{\!\!\mathsf i}}}
\newcommand{\Ibshift}{\mathbin{\swarrow_{\!\!\mathsf i}}}

\newcommand{\shift}[2]{#1\Rshift #2}
\newcommand{\bshift}[2]{#1\Rbshift #2}

\newcommand{\dshift}[2]{#1{\downarrow_{#2}}}

\newcommand{\pb}{\mathord{\downarrow}}
\newcommand{\po}{\mathord{\uparrow}}

% Letters and variables
\newcommand{\Lab}{\mathsf{L}}

\newcommand{\AB}[1]{\mathsf{AB}\if\relax\detokenize{#1}\relax\else(#1)\fi}
\newcommand{\AC}[1]{\mathsf{AC}\if\relax\detokenize{#1}\relax\else(#1)\fi}

\newcommand{\SB}[1]{\mathsf{SB}\if\relax\detokenize{#1}\relax\else(#1)\fi}
\newcommand{\SC}[1]{\mathsf{SC}\if\relax\detokenize{#1}\relax\else(#1)\fi}

\newcommand{\bC}{\mathbf{C}}
\newcommand{\bAC}{\mathbf{AC}}

\newcommand{\cA}{\mathcal{A}}
\newcommand{\ccA}{\cA^\circ}
\newcommand{\cB}{\mathcal{B}}
\newcommand{\cC}{\mathcal{C}}
\newcommand{\cE}{\mathcal{E}}
\newcommand{\cF}{\mathcal{F}}
\newcommand{\cM}{\mathcal{S}}
\newcommand{\cN}{\mathcal{N}}
\newcommand{\cI}{\mathcal{I}}
\newcommand{\cJ}{\mathcal{J}}
\newcommand{\cP}{\mathcal{P}}
\renewcommand{\cR}{\mathcal{R}}
\newcommand{\cS}{\mathcal{S}}
\newcommand{\cT}{\mathcal{T}}
\newcommand{\cU}{\mathcal{U}}
\newcommand{\cV}{\mathcal{V}}

% Direct forward-shifting
\newcommand{\dF}[1]{\cF^\circ_{#1}}
% Retraction-based forward-shifting
\newcommand{\rF}[2]{\cF^{#2}_{#1}}
% Direct backward-shifting
\newcommand{\dB}[1]{\cB^\circ_{#1}}
% Retraction-based backward-shifting
\newcommand{\rB}[2]{\cB^{#2}_{#1}}
\newcommand{\ccP}{\cP^\circ}

\newcommand{\forw}{{\mathsf{f}}}
\newcommand{\back}{{\mathsf{b}}}

% \newcommand{\ABC}{\cat{AB-Cond}}
\newcommand{\ABC}{\mathbf{AC}}
\newcommand{\SBC}{\mathbf{SC}}

\newcommand{\Hom}{\mathrm{Hom}}

\newcommand{\Hyper}{\cat{Hyper}}
\newcommand{\Graph}{\cat{Graph}}
\newcommand{\Cospan}{\cat{Cospan}}
\newcommand{\SpanC}{\cat{Span}(\cat{C})}


\newcommand{\False}{\textbf{false}}
\newcommand{\True}{\textbf{true}}

% Alignment macro, see https://www.math.lsu.edu/~aperlis/publications/mathclap/perlis_mathclap_24Jun2003.pdf
\def\clap#1{\hbox to 0pt{\hss#1\hss}}

% Graph labels = binary relations
% Graph label font
\newcommand{\gl}[1]{\mathsf{#1}}

\newcommand{\la}{\gl{a}}
\newcommand{\lb}{\gl{b}}
\newcommand{\lc}{\gl{c}}
\newcommand{\ld}{\gl{d}}

\newcommand{\mapping}[1]{%
  \scalebox{.7}{$\begin{array}{@{}r@{\mapsto}l@{}}
  #1
  \end{array}$}
}

% Auxiliary labelling macros
% For definitions
\newcommand{\dprf}{def}
\newcommand{\dlabel}[1]{\label{\dprf:#1}}
\newcommand{\dref}[1]{\ref{\dprf:#1}}
\newcommand{\dcite}[1]{Def.~\dref{#1}}

% For figures
\newcommand{\fprf}{fig}
\newcommand{\flabel}[1]{\label{\fprf:#1}}
\newcommand{\fref}[1]{\ref{\fprf:#1}}
\newcommand{\fcite}[1]{Fig.~\fref{#1}}

% For propositions
\newcommand{\pprf}{prop}
\newcommand{\plabel}[1]{\label{\pprf:#1}}
\newcommand{\pref}[1]{\ref{\pprf:#1}}
\newcommand{\pcite}[1]{Prop.~\pref{#1}}

% For lemmas
\newcommand{\lprf}{lem}
\newcommand{\llabel}[1]{\label{\lprf:#1}}
\newcommand{\lref}[1]{\ref{\lprf:#1}}
\newcommand{\lcite}[1]{Lemma~\lref{#1}}

% For sections
\newcommand{\sprf}{sec}
\newcommand{\slabel}[1]{\label{\sprf:#1}}
\newcommand{\sref}[1]{\ref{\sprf:#1}}
\newcommand{\scite}[1]{Sec.~\sref{#1}}

% For examples
\newcommand{\exprf}{ex}
\newcommand{\exlabel}[1]{\label{\exprf:#1}}
\newcommand{\exref}[1]{\ref{\exprf:#1}}
\newcommand{\excite}[1]{Ex.~\exref{#1}}

% For assumptions
\newtheorem{assumption}{Assumption}
\newcommand{\assprf}{ass}
\newcommand{\asslabel}[1]{\label{\assprf:#1}}
\newcommand{\assref}[1]{\ref{\assprf:#1}}
\newcommand{\asscite}[1]{Assumption~\assref{#1}}

% For equations
\newcommand{\eqprf}{eq}
\newcommand{\eqlabel}[1]{\label{\eqprf:#1}}
\newcommand{\eqcite}[1]{Eq.~\eqref{\eqprf:#1}}

%% Tikz definitions
\usetikzlibrary{shapes.geometric}
\usetikzlibrary{positioning}
\usetikzlibrary{arrows.meta}
\usetikzlibrary{calc}

\tikzset{>=latex}
\tikzset{triangle/.style=
  {isosceles triangle,
   draw,
   shape border rotate=90,
   isosceles triangle stretches=true,
   anchor=top corner,
   minimum width=15mm,
   minimum height=15mm}}
\tikzset{graph/.style={
   inner sep=1pt,
   line width=3pt,
   draw=white,
   double=black}}
\tikzset{morphism/.style={
   draw=brown!50!black,color=brown!50!black}}
\tikzset{over/.style={
   line width=.8pt,
   double distance=0.4pt,
   double=black,
   draw=white,
   arrows={|-|},
   >/.tip={Latex[color=black,scale=.5]}
}}
\tikzset{allcolor/.style= {
   draw={#1},
   text={#1},
   >/.tip={latex[color={#1}]}
}}

\tikzset{|/.tip={Bar[black,width=.4pt,line width=2pt]}}
\tikzset{cross/.style={
   preaction={draw=white,line width=2pt}
}}

% #1 - colour (optional)
% #2 - name of top node
% #3 - height
% #4 - offset of right bottom point
% #5 - (bottom) width
% #6 - label
\newcommand{\tri}[6][black]{
  \path [fill=#1!10,draw=#1,opacity=.5]
        (#2)
		coordinate (#2-top)
		-- +($(#4,-#3)$)
		coordinate (#2-right)
		-- +($(#4,-#3)-(#5,0)$)
		coordinate (#2-left)
		-- (#2);
  \path (#2)
        edge[draw=none]
		node[near end,text=#1] (#2-label) {#6}
		+($(#4,-#3)-(#5/2,0)$);
}

\newcommand{\darker}[1]{#1!50!black}

\newcommand{\mygraph}[1]{%
\renewcommand{\gl}[1]{\ensuremath{\scriptstyle{\mathsf{##1}}}}%
\tikzset{n/.style={inner sep=1pt}}%
\begin{tikzpicture}[on grid,scale=.5,inner sep=2pt,baseline=(current bounding box.center)]
\begin{scope}[node distance=8mm]
#1
\end{scope}
\end{tikzpicture}}

% Discrete one-node graph
% #1: node
\newcommand{\onenode}[1]{\mygraph{
\node[n] (1) {$#1$};
}}

% Discrete two-node graph
% #1: node
\newcommand{\twonode}[2]{\mygraph{
\node[n] (1) {$#1$};
\node[n] (2) [right=.5 of 1] {$#2$};
}}

% One-node graph with loop
% #1: node
% #2: loop label
\newcommand{\oneloop}[2]{\mygraph{
\node[n] (1) {$#1$};
\path (1) edge [->,loop right] node {\gl{#2}} ();
}}

% two-node graph with loop
% #1: first node
% #2: right-pointing edge
% #1: second node
% #2: left-pointing edge
\newcommand{\twoloop}[4]{\mygraph{
\node[n] (1) {$#1$};
\node[n] (2) [right=of 1] {$#3$};
\path (1) edge [->,bend left] node[near start,above] {\gl{#2}} (2)
      (2) edge [->,bend left] node[near start,below] {\gl{#4}} (1);
}}

% One-edge graph
% #1: source node
% #2: edge label
% #3: target node
\newcommand{\oneedge}[3]{\mygraph{
\node[n] (1) {$#1$};
\node[n] (2) [right=of 1] {$#3$};
\path (1) edge[->] node[above] {\gl{#2}} (2);
}}

% Edge + loop graph
% #1: source node
% #2: edge label
% #3: target node
% #4: loop label
\newcommand{\oneedgeloop}[4]{\mygraph{
\node[n] (1) {$#1$};
\node[n] (2) [right=of 1] {$#3$};
\path (1) edge[->] node[above] {\gl{#2}} (2)
      (2) edge [->,loop right] node {\gl{#4}} ();
}}

% Loop + edge graph
% #1: source node
% #2: loop label
% #3: edge label
% #4: target node
\newcommand{\looponeedge}[4]{\mygraph{
\node[n] (1) {$#1$};
\node[n] (2) [right=of 1] {$#4$};
\path (1) edge[->] node[above] {\gl{#3}} (2)
      (1) edge [->,loop left] node {\gl{#2}} ();
}}

% Serial two-edge graph
% #1: first node
% #2: first edge label
% #3: middle node
% #4: second edge label
% #5: third node
\newcommand{\twoedge}[5]{\mygraph{
\node[n] (1) {$#1$};
\node[n] (2) [right=of 1] {$#3$};
\node[n] (3) [right=of 2] {$#5$};
\path (1) edge[->] node[above] {\gl{#2}} (2)
      (2) edge[->] node[above] {\gl{#4}} (3);
}}

% One edge followed by a split into two edges
% #1: first node
% #2: first edge label
% #3: middle node
% #4: second edge label
% #5: upper right node
% #6: third edge label
% #7: lower right node
\newcommand{\onetwoedge}[7]{\mygraph{
\node[n] (1) {$#1$};
\node[n] (2) [right=of 1] {$#3$};
\node[n] (3) [above right=.3 and 1 of 2] {$#5$};
\node[n] (4) [below right=.3 and 1 of 2] {$#7$};
\path (1) edge[->] node[above] {\gl{#2}} (2)
      (2) edge[->] node[above] {\gl{#4}} (3)
      (2) edge[->] node[below] {\gl{#6}} (4);
}}

% Span graph
% #1: first node
% #2: first edge label
% #3: joint target node
% #4: second edge label
% #5: third node
\newcommand{\spangraph}[5]{\mygraph{
\node[n] (1) {$#1$};
\node[n] (2) [right=of 1] {$#3$};
\node[n] (3) [right=of 2] {$#5$};
\path (1) edge[<-] node[above] {\gl{#2}} (2)
      (2) edge[->] node[above] {\gl{#4}} (3);
}}

% Cospan graph
% #1: first node
% #2: first edge label
% #3: joint target node
% #4: second edge label
% #5: third node
\newcommand{\cospangraph}[5]{\mygraph{
\node[n] (1) {$#1$};
\node[n] (2) [right=of 1] {$#3$};
\node[n] (3) [right=of 2] {$#5$};
\path (1) edge[->] node[above] {\gl{#2}} (2)
      (2) edge[<-] node[above] {\gl{#4}} (3);
}}
