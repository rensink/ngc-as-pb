\subsection{Related work}
\slabel{related}

In his work on Existential Graphs~\cite{roberts1973-the-existential-graphs-of-charles-s.-peirce}, Charles S. Peirce proposed a graphical representation of full FOL, equipped with some kinds of graph manipulations which represent sound deductions. Even if equally expressive, comparing Existential Graphs with Nested Conditions looks difficult, because the former are not formalized in a standard algebraic/categorical way.

Bonchi et.al.~\cite{DBLP:conf/csl/BonchiSS18} have enriched the correspondence between graphs and conjunctive queries of~\cite{DBLP:conf/stoc/ChandraM77}, summarized in the Introduction,  by identifying a common rich categorical structure: \emph{cartesian bicategories}. They introduce \emph{graphical conjunctive queries} as suitable \emph{string diagrams}, i.e.~arrows of a specific free cartesian bicategory, showing that they are as expressive as standard conjunctive queries and, more interstingly, that the freely generated preorder among them is exactly the entailment preoder among queries. Furthermore, exactly the same algebraic structure is show to arises by considering as arrows cospans of hypergraphs and as preorder the existence of a morphism. This is summarized by a triangular relationship including logical structures (queries), combinatorial structures (hypergraphs), and categorical ones (free cartesian bicategories).  This characterization of conjunctive formulas as arrows of a free cartesian bicategory has been generalized in~\cite{DBLP:journals/corr/abs-2404-18795} to full FOL, but lacking a combinatorial/graphical counterpart of formulas. As a possible development of the results of this paper, by equipping conditions with suitable interfaces (encoding free variables) we intend to study the algebraic structure of conditions, possibly identifying a suitable cartesian bicategory.  This could provide the third missing structure (the combinatorial one), allowing to lift to full FOL the triangular correspondence presented in \cite{DBLP:conf/csl/BonchiSS18} for the $\exists$-fragment.


The operation of shifting a condition along an arrow has been exploited for cospan-based conditions in~\cite{bchk:conditional-reactive-systems} to compute weakest preconditions and strongest postconditions for graph transformation systems. This operation, similarly to our notion of shifter, defines a functor between catergories of conditions that is shown to have both a left and right adjoint, corresponding to a form of existential and universal quantification.  We intend to define logical operations on span-based conditions, and to explore whether the adjunction results still hold.

In~\cite{bchk:conditional-reactive-systems,sksclo:coinductive-techniques-for-satisfiability} nested conditions are defined over an arbitrary category $\bC$, which allows to instantiate the framework beyond presheaf toposes: for example, they consider the category of graphs and injective morphisms, or the category of left-linear cospans of an adhesive category. \todo{AR: in what sense?} To which our framework cannot be applied.

Some papers~\cite{lo:tableau-graph-properties,slo:model-generation,sksclo:coinductive-techniques-for-satisfiability} address the problem of (semi-)deciding satisfiability of nested conditions, by resorting to tableau-based techniques inspired by those of FOL. The proposed algorithms are also able to generare finite models if a formula has one.
It would be interesting to explore if passing from arrow-based to span-based conditions can have an impact on  the complexitiy of proofs, and if morphisms among conditions could allow to relate different tableau based proofs.
