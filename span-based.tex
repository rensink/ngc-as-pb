\section{Span-based conditions}

We now present the main idea of this paper, inspired by on the observation that arrow-based conditions over graphs contain a lot of duplication: the children of a node essentially contain all the structure of that (parent) node, and add some constraints on top of it. It turns out to be possible to avoid this duplication by only specifying the \emph{additional} structure (and how it is connected). This is achieved by replacing the branch arrows $r^c_i$ of a condition $c$ by \emph{spans}.\footnote{We discuss and contrast an alternative way to resolve the same issue, using \emph{cospans} rather than spans, later in this paper.}

Span-based conditions are inductively defined as follows:

\begin{definition}[span-based condition]\dlabel{sb-condition}
  For any object $R$ of $\bC$, $\SC R$ (the set of \emph{span-based conditions} over $R$) and $\SB R$ (the set of \emph{span-based branches} over $R$) are the smallest sets such that
  \begin{itemize}
  \item $c\in \SC R$ if $c=(R,p_1\ccdots p_w)$ is a pair with $p_i\in \SB R$ for all $1\leq i\leq w$;
  \item $p\in \SB R$ if $p=(\spanof u d,c)$ where $u: I\to R, d:I\func P$ form a span of arrows of $\bC$ and $c\in \SC P$.
  \end{itemize}
\end{definition}
%
In a branch $p=(\spanof u d,c)$, $u$ stands for the \emph{up-arrow} and $d$ for the \emph{down-arrow}. As before, we use $|c|=w$ to denote the width of a span-based condition $c$, $R^c$ to denote its root, and $p^c_i=(s^c_i,c_i)$ with span $s^c_i=\spanof{u^c_i}{d^c_i}$ its $i$-th branch. Finally, we use $I^c_i$ for the interface of span $s^c_i$ and $P^c_i$ ($=R^{c_i}$) for its target (hence $u^c_i:I^c_i\func R^c$ and $d^c_i:I^c_i\rightarrow P^c_i$). In all these cases, we may omit the superscript $c$ if it is clear from the context. Pictorially, $c$ can be visualised as in \fcite{sb-condition}.
%
\begin{figure}
  \centering
  \begin{tikzpicture}[on grid]
  \node (R) {$R$};
  \node (I1) [below left=of R] {$I_1$};
  \node (R1) [below=1.2 of I1] {$R_1$};
  \node (c1) [triangle,below=.15 of R1.center] {$c_1$};
  \node [below=.8 of Rc] {$\cdots$};
  \node (In) [below right=of R] {$I_n$};
  \node (Rn) [below=1.2 of In] {$R_n$};
  \node (cn) [triangle,below=.15 of Rn.center] {$c_n$};

  \path (I1) edge[->] node[above left] {$u_1$} (R)
        (I1) edge[->] node[left] {$d_1$} (R1)
        (In) edge[->] node[above right] {$u_n$} (R)
        (In) edge[->] node[right] {$d_n$} (Rn);
\end{tikzpicture}

  \caption{Pictorial representation of a span-based condition $c=(R,p_1\ccdots p_w)$, with $p_i=(\spanof{u_i}{d_i},c_i)$ for all $1\leq i\leq w$}
  \flabel{sb-condition}
\end{figure}
%
As we will see later, we can reconstruct an arrow-based condition from a span-based one by taking pushouts over the spans $\spanof{u_i}{d_i}$; the arrows $a_i$ will then correspond to the image of $d_i$ in the pushout cospan.

\medskip\noindent We now present the modified notion of satisfaction for span-based conditions. First we recognise that the purpose of an arrow $a$ in an ab-condition is essentially to establish ``correct" model/witness pairs $(g,h)$ --- namely, those pairs that commute with $a$ in the sense of satisfying $g=a;h$. We may write $a\commutes (g,h)$ to express this commutation relation. This is what we now formalise for spans, as follows:
\[ \spanof u d \commutes (g,h) \text{ if } u;g=d;h \enspace. \]
Satisfaction of sb-conditions is then defined as follows:

\begin{definition}[satisfaction of span-based conditions]\dlabel{sb-satisfaction}
  Let $c$ be a span-based condition and $g:R\func G$ an arrow from $c$'s root to an object $G$. We say that \emph{$g$ satisfies $c$}, denoted $g\sat c$, if there is a branch $p_i$ and an arrow $h:R_i\func G$ such that
  \begin{enumerate*}
  \item $s_i \commutes (g h)$, and
  \item $h\nsat c_i$.
  \end{enumerate*}
\end{definition}
%
Note that this is entirely analogous fo \dcite{ab-satisfaction}, especially if we would retrofit the notation $a\commutes (g,h)$ there. Like before, we call $p_i$ the \emph{responsible branch} and $h$ the \emph{witness} of $g\sat c$. Satisfaction of sb-conditions is visualised in \fcite{sb-satisfaction}.
%
\begin{figure}
  \centering
  \begin{tikzpicture}
  \node (Rc) {$R_c$};
  \node (I1) [below left=.3 and .2 of Rc.center] {$\cdots$};
  \node (Ip) [below=of Rc] {$I_p$};
  \node (In) [below right=.3 and .2 of Rc.center] {$\cdots$};
  \node (Rp) [below=.7 of Ip] {$R_{c_p}$};
  \node (cp) [triangle,below=.15 of Rp.center] {$c_p$};
  \node (G) [right=2 of Rc] {$G$};

  \path (Rc) edge[->] node[above] {$g$} (G)
        (Ip) edge[->] node[left] {$u_p$} (Rc)
        (Ip) edge[->] node[left] {$d_p$} (Rp)
        (Rp) edge[->] node[pos=0.3,below right] (h) {$h$} (G)
        (h) edge[draw=none] node[sloped,allow upside down] {$\nsat$} (cp);
\end{tikzpicture}

  \caption{Pictorial representation of $g\sat c$ for a span-based condition $c$, with responsible branch $p_i=(\spanof{u_i}{d_i},c_i)$ and witness $h$ such that $\spanof{u_i}{d_i}\commutes (g,h)$}
  \flabel{sb-satisfaction}
\end{figure}

\subsection{Span-based root shifting}

To define morphisms of sb-conditions, we will go over the same ground as for ab-conditions, but take a slightly more abstract view --- as we did for satisfaction by introducting $\commutes$. Having recognised that the \emph{semantics} of a span in sb-conditions is essentially given by the set of model/witness pairs that it commutes with, we can now in general consider relations over spans that either preserve or reflect their semantics, in the sense that if $s_1,s_2$ are related, then either $s_1\commutes (g,h)$ implies $s_2\commutes (g,h)$ (preservation) or $s_2\commutes (g,h)$ implies $s_1\commutes (g,h)$ (reflection). In most cases, however, the model/witness pairs are not preserved or reflected precisely, but modulo some arrow that gets added to or erased from the model or witness. Thus:\todo{shorten}
%
\begin{itemize}
\item $(s_1,s_2)$ preserves models plus $v$ if $s_1\commutes (g,h)$ implies $s_2\commutes (v;g,h)$;
\item $(s_1,s_2)$ preserves models minus $v$ if $s_1\commutes (v;g,h)$ implies $s_2\commutes (g,h)$;
\item $(s_1,s_2)$ reflects models plus $v$ if $s_2\commutes (g,h)$ implies $s_1\commutes (v;g,h)$;
\item $(s_1,s_2)$ reflects models minus $v$ if $s_2\commutes (v;g,h)$ implies $s_1\commutes (g,h)$;
\item $(s_1,s_2)$ preserves witnesses plus $v$ if $s_1\commutes (g,h)$ implies $s_2\commutes (g,v;h)$;
\item $(s_1,s_2)$ preserves witnesses minus $v$ if $s_1\commutes (g,v;h)$ implies $s_2\commutes (g,h)$;
\item $(s_1,s_2)$ reflects witnesses plus $v$ if $s_2\commutes (g,h)$ implies $s_1\commutes (g,v;h)$;
\item $(s_1,s_2)$ reflects witnesses minus $v$ if $s_2\commutes (g,v;h)$ implies $s_1\commutes (g,h)$;
\end{itemize}
%
The above is defined on individual pairs. For a relation $\cR$ over spans, we say that $\cR$ preserves [reflects] models [witnesses] plus [minus] $v$ if it consists of pairs that do so.

To extend \eqcite{shifters are functors} from arrows to spans, we also need to consider span composition. This is defined as usual: $\spanof u d;\spanof v w=\spanof{v';u}{d';w}$ where $(v',d')$ is the pullback span of $(d,v)$.\footnote{Technically, this is only well-defined if we consider spans up to isomorphism of their middle objects; we will gloss over this here.}

\begin{definition}[span source shifters]
Let $X,Y$ be objects. A \emph{span source shifter} $\cS$ from $X$ to $Y$ is a function from $X$-sourced spans to $Y$-sourced spans having the same target, say $Z$, such that for all spans $s:X\to Y,t:X\to Z$:
%
\begin{equation}\eqlabel{span shifters are functors}
\cS(s;t) = \cS(s);t \enspace.
\end{equation}
%
Now let $v=A\to B$ be an arrow. $\cS$ is a \emph{forward span source shifter for $v$} if it is a span source shifter from $A$ to $B$ that preserves models minus $v$, i.e., such that for all spans $s:A\to C$ and all and all pairs $g:A\to G,h:C \to G$:
%
\begin{equation}\eqlabel{sb-forward condition}
s\commutes (v;g,h) \enspace\text{ implies }\enspace \cS(s) \commutes (g,h) \enspace.
\end{equation}
%
A forward span source shifter is called \emph{complete}\todo{other term?} if it also reflects models plus $v$; i.e., if the ``implies" in \eqref{eq:forward condition} is an `` if and only if".

Instead, $\cS$ is a \emph{backward span source shifter for $v$} if it is a span source shifter from $B$ to $A$ that reflects models minus $v$, i.e., such that for all spans $s:B\to C$ and all pairs $g:B\to G,h:C\to G$:
%
\begin{equation}\eqlabel{sb-backward condition}
\cS(s) \commutes (v;g,h) \enspace\text{ implies }\enspace s\commutes (g,h) \enspace.
\end{equation}
%
A backward span source shifter is called \emph{complete} if it also preserves models plus $v$; i.e., if the ``implies" in \eqref{eq:backward condition} is an `` if and only if".
\end{definition}
%
Span source shifters extend to root shifters (of span-based conditions) exactly as in the arrow-based case. Before showing how this gives rise to morphisms, however, let us investigate the possible instantiations of span source shifters.



\subsection{Span-based morphisms}

A final change with respect to the arrow-based case is that we also abstract the commutation condition in \dcite{ab-morphism} by relying instead on a family or relations $\cR_v$ over spans (where $m$ is an arrow in $\bC$) such that $\cR_v$ reflects witnesses plus $v$. $\cR_v$ is called \emph{complete} if it also preserves witnesses minus $m$. To see the analogy with the arrow-based case, note that $\cR_v=\setof{(a_1,a_2)\mid a_1;v=a_2}$ precisely satisfies the requirement of reflecting witnesses plus $v$ (and if $v$ is iso, $R_v$ is complete, though we did not consider this notion for ab-morphisms).
%
\begin{definition}[span-based (complete) condition morphism]\dlabel{sb-morphism}
  Given two sb-conditions $b,c$, a \emph{forward-shif [backward-shift] sb-morphism} $m:b\to c$ is a pair $(o,(v_1,m_1)\ccdots (v_{|b|},m_{|b|}))$ where
  \begin{itemize}
  \item $o:[1,|b|]\to [1,|c|]$ is a function from $b$'s branches to $c$'s branches;
  \item for all $1\leq i\leq |b|$, $v_i:P^c_{o(i)}\to P^b_i$ is a mapping from the pattern of $p^c_{o(i)}$ to that of $p^b_i$ such that $(s^b_i,s^c_{o(i)})\in \cR_{v_i}$;
  \item \emph{Forward shift:} for all $1\leq i\leq |b|$, there is a forward root shifter $F_i$ for $v_i$ such that $m_i:\cF_i(c_{o(i)})\to b_i$ is a forward-shift morphism;
  \item \emph{Backward shift:} for all $1\leq i\leq |b|$, there is a backward root shifter $\cB_i$ for $v_i$ such that $m_i:c_{o(i)}\to \cB(b_i)$ is a backward-shift morphism.
  \end{itemize}
  $m$ is called \emph{complete} if $o$ is surjective and for all $1\leq i\leq |b|$, $\cR_{v_i}$ is complete and $\cF_i$ [$\cB_i$] is complete.
\end{definition}
%
As before, we want morphisms to satisfy certain ``nice" properties: preservation of semantics, reflection in case of completeness, and compositionality. As it turns out, these all follow from the nature of forward and backward source shifters in combination with that of $\cR_v$. However, before embarking on the proofs, we will first give an concrete instance of $\cR_v$.


%
\begin{center}
\begin{tikzpicture}[on grid]
  \node (Rb) [below] {$R^b$};
  \node (Ib) [below=of Rb] {$I^b_i$};
  \node (Pb) [below=of Ib] {$P^b_i$};
  \node (cb) [triangle,below=.15 of Pb.center] {$b_i$};

  \path (Ib) edge[->] node[left] {$u^b_i$} (Rb)
        (Ib) edge[->] node[left] {$d^b_i$} (Pb);

  \node (Rc) [right=2.5 of Rb] {$R^c$};
  \node (Ic) [below=of Rc] {$I^c_{o(i)}$};
  \node (Pc) [below=of Ic] {$P^c_{o(i)}$};
  \node (cc) [triangle,below=.15 of Pc.center] {$c_{o(i)}$};

  \path (Ic) edge[->] node[right] {$u^c_{o(i)}$} (Rc)
        (Ic) edge[->] node[right] {$d^c_{o(i)}$} (Pc);

  \path (Rb) edge[->] node[above] {$t^m$} (Rc)
        (Ic) edge[->] node[above] {$k_i$} (Ib)
        (Pc) edge[->] node[above] {$t_i$} (Pb);
\end{tikzpicture}

\end{center}
%
We write $t^m$ and $o^m$ for the components of $m$, and (for all $1\leq i\leq |b|$) $t^m_i$ ($=t_{m_i}$) for the top-level arrow of $m_i$ --- often omitting the superscript $m$ if it is clear from the context. Pictorially, $m$ can be visualised as in \fcite{sb-morphism}.
%
\begin{figure}
  \centering
  \begin{tikzpicture}[>=latex,on grid]
  \node (Rb) {$R_b$};
  \node (I1) [below left=1.5 and 1 of Rb] {$I^b_1$};
  \node (R1) [below=1.5 of I1] {$R_{b_1}$};
  \node (b1) [triangle,below=.12 of R1.center] {$b_1$};
  \node [below=1 of Rb] {$\cdots$};
  \node (In) [below right=1.5 and 1 of Rb] {$I^b_n$};
  \node (Rn) [below=1.5 of In] {$R_{b_n}$};
  \node (bn) [triangle,below=.12 of Rn.center] {$b_n$};

  \path (I1) edge[->] node[left] {$u^b_1$} (Rb)
        (I1) edge[->] node[left] {$d^b_1$} (R1)
		(In) edge[->] node[right] {$u^b_n$} (Rb)
        (In) edge[->] node[right,near end] {$d^b_n$} (Rn);

  \node (Rc) [right=7 of Rb] {$R_c$};
  \node [below left=1 and 1.5 of Rc] {$\cdots$};
  \node (Ji) [below left=1.5 and 1 of Rc] {$I^c_i$};
  \node (Ri) [below=1.5 of Ji] {$R_{c_i}$};
  \node (ci) [triangle,below=.12 of Ri.center] {$c_i$};
  \node [below=1 of Rc] {$\cdots$};
  \node (Jj) [below right=1.5 and 1 of Rc] {$I^c_j$};
  \node (Rj) [below=1.5 of Jj] {$R_{c_j}$};
  \node (cj) [triangle,below=.12 of Rj.center] {$c_j$};
  \node [below right=1 and 1.5 of Rc] {$\cdots$};

  \path (Ji) edge[->] node[left] {$u^c_i$} (Rc)
        (Ji) edge[->] node[left,near end] {$d^c_i$} (Ri)
		(Jj) edge[->] node[right] {$u^c_j$} (Rc)
        (Jj) edge[->] node[right] {$d^c_j$} (Rj);

  \path (Rb) edge[->] node[above] {$t_m$} (Rc)
        (Ji) edge[over,|->,dashed,bend left=20] node[above,very near start] {$k_1$} (I1)
        (Ri) edge[over,|->,bend left=20] node[above,very near start] {$t_{m_1}$} (R1)
        (ci) edge[over,|->,bend left=20] node[below,very near start] {$m_1$} (b1)
        (Jj) edge[over,|->,dashed,bend left=20] node[below,very near start] {$k_n$} (In)
        (Rj) edge[over,|->,bend left=20] node[below,very near start] {$t_{m_n}$} (Rn)
        (cj) edge[over,|->,bend left=20] node[below,very near start] {$m_n$} (bn);
\end{tikzpicture}

  \caption{Pictorial representation of a span-based condition morphism $m:b\func c$ (with $w=|b|$ and $m=(t,o,m_1\ccdots m_w)$)}
  \flabel{sb-morphism}
\end{figure}
%
%\medskip\noindent
%\emph{Note to self: I find it surprising that there are no conditions on $k_p$ except for the confluence equations. For instance, $k_p,d_q$ is \emph{not} necessarily a pullback of $t_{m_p},d_p$ (\excite{no pullback} below); in fact, $k_p$ is \emph{not even} uniquely determined by the two confluence equations (\excite{k not unique} below). This is one of the reasons why the $k$ are not part of the morphism, but merely required to exist; otherwise there can be multiple morphisms between the same conditions that only differ in their $k$-components.}

\medskip\noindent Again, span-based condition morphisms have the expected properties: identities and composition exist and the categorical laws are satisfied, thus span-based conditions and their morphisms form a category.

\begin{proposition}[category \cat{SB-Cond}]
  The category $\cat{SB-Cond}$ having span-based conditions (\dcite{sb-condition}) as objects and span-based condition morphisms (\dcite{sb-morphism}) as arrows is well-defined. In addition, identities in $\cat{SB-Cond}$ are complete, and complete morphisms compose.
\end{proposition}
 
\begin{proof}
  Given an sb-condition $c=(R,p_1\ccdots p_w)$, the identity morphism $\id_c:c\func c$ is inductively defined as $(\id_R,\id_w,\id_{c_1}\ccdots \id_{c_w})$: the required commutativity properties trivially hold by choosing $k_i=\id_{I_i}$. By construction, $\id_c$ is complete.
  
Given two ab-condition morphisms $m:b\func c$ and $n:c\func e$, their composition is defined as $m;n=(t^m;t^n,o^m;o^n,(n_{o^m(1)};m_1)\ccdots (n_{o^m(w)};m_{w}))$ (where $w=|b|$).
The well-definedness of $m;n$ can be proved in a standard way, choosing as intermediate morphisms $k_i=k^n_{o(i)};k^m_i$ for all $1\leq i\leq w$. If $m$ and $n$ are complete, so is $m;n$ (due to the fact that $o^m;o^n$ is surjective if both $o^m$ and $o^n$ are, and that pushouts compose).

The identity laws and associativity of composition can be proved in a standard way. \qed
\end{proof}
%
With this notion of morphism, we not only have the desired preservation of satisfaction, but also reflection of satisfaction in case of completeness, in the following sense:
%
\begin{proposition}[morphisms preserve satisfaction]\plabel{sb-morphism preserves satisfaction}
Let $b$ and $c$ be span-based conditions. If $m:b\func c$ is an sb-condition morphism, then $t;g\sat b$ implies $g\sat c$ for all arrows $g:R^c\to G$. Moreover, if $m$ is complete then also the converse is true: $g\sat c$ implies for $t;g\sat b$ all arrows $g:R^c\to G$.
\end{proposition}
%
\begin{proof}
By induction. Assume that $p_i$ is the responsible branch of $b$ and $h^b\colon R_{c_p}\func G$ the witness for $t;g\sat b$; hence the induction hypothesis implies that the proposition holds for $c_{o(i)}$, and moreover, $u^b_i;t;g=d^b_i;h^b$ and $h^b\nsat b_i$. Moreover, let $j=o(i)$ and let $k\colon I^c_j\func I^b_i$ be as required in \dcite{sb-morphism}. This is visualised in \fcite{sb-morphism preserves satisfaction}.
%
\begin{figure}
  \centering
  \begin{tikzpicture}[on grid]
  \node (G) {$G$};
  
  \node (Rb) [below left=1.5 and 2 of G] {$R^b$};
  \node (Ib1) [below left=of Rb] {};
  \node [below left=.8 and .4 of Rb] {$\cdots$};
  \node (Ibi) [below=1.8 of Rb] {$I^b_i$};
  \node [below right=.8 and .4 of Rb] {$\cdots$};
  \node (Ibn) [below right=of Rb] {};
  \node (Ri) [below=1.5 of Ibi] {$R^b_i$};
  \node (bi) [triangle,below=.15 of Ri.center] {$b_i$};

  \path (Ib1) edge[->] (Rb)
		(Ibn) edge[->] (Rb) 
		(Ibi) edge[->] node[right,near start] {$u^b_i$} (Rb)
        (Ibi) edge[->] node[right] {$d^b_i$} (Ri);

  \node (Rc) [below right=1.5 and 2 of G] {$R^c$};
  \node (Ic1) [below left=of Rc] {};
  \node [below left=.8 and .4 of Rc] {$\cdots$};
  \node (Icj) [below=1.8 of Rc] {$I^c_j$};
  \node [below right=.8 and .4 of Rc] {$\cdots$};
  \node (Icn) [below right=of Rc] {};
  \node (Rj) [below=1.5 of Icj] {$R^c_j$};
  \node (cj) [triangle,below=.15 of Rj.center] {$c_j$};

  \path (Ic1) edge[->] (Rc)
		(Icn) edge[->] (Rc) 
		(Icj) edge[->] node[left,near start] {$u^c_j$} (Rc)
        (Icj) edge[->] node[left] {$d^c_j$} (Rj);

  \path (Rb) edge[->] node[above] {$t$} (Rc)
        (Icj) edge[->] node[above] {$k_i$} (Ibi)
        (Rj) edge[->] node[above] {$t_i$} (Ri)
        (cj) edge[->] node[above] {$m_i$} (bi);

  \path (Rc) edge[->] node[above right] {$g$} (G)
        (Ri) edge[->,bend left=75,min distance=3cm] node[near start,left] {$h^b$} (G) 
        (Rj) edge[->,bend right=75,min distance=3cm,color=gray] node[near start,right] {$h^c$} (G);  
\end{tikzpicture}

  \caption{Pictorial representation of \pref{sb-morphism preserves satisfaction}: if $m:b\rightarrow c$ is an sb-morphism, then $t;g\sat b$ implies $g\sat c$}
  \flabel{sb-morphism preserves satisfaction}
\end{figure}

Now $h^c=t_i;h^b$ can be shown to satisfy the conditions for $h$ in \dcite{sb-satisfaction} to be a witness for $g\sat c$, with responsible branch $p^c_j$:
\begin{enumerate}
\item $u^c_j;g = k_i;u^c_i;t;g= k_i;d^b_i;h^b = d^c_j;t_i;h^b=d^c_j;h^c$.
\item Suppose that $h^c\sat c_j$, meaning $t_i;h^b\sat c_j$; then by the induction hypothesis and the fact that $m_i$ is a morphism from $c_j$ to $b_i$, it follows that $h^b\sat b_i$, which is in contradiction with the initial assumption of this proof. Hence $h^c\nsat c_j$.
\end{enumerate}
%
Now assume that $m$ is complete and $g\sat c$, with responsible branch $p^c_j$ and witness $h^c$. The essential observation is that we can find a responsible branch $p^b_i$ and witness $h^c$ showing $t;g\sat b$.

In fact, since $o$ is surjective, take $i$ such that $j=o(i)$; this determines $p^b_i$. Now observe that $k_i;u^b_i;t;g=u^c_j;g=d^c_j;h^c$ and hence (since the lower square of \fcite{sb-morphism preserves satisfaction} is a pushout) there is a unique $h^b:P^b_i\to G$ with $d^b_i;h^b=u^b_i;t;g$ and $t_i;h^b=h^c$. Together with the induction hypothesis and the fact that $m^i$ is complete, these are precisely the properties needed for $h^b$ to be a witness of $t;g\sat b$.\qed
\end{proof}

\subsection{From arrow-based to span-based conditions}

Span-based conditions are richer than arrow-based conditions. In this subsection we show the existence of a full, satisfaction-invariant functor from $\ABC_\bC$ to $\SBC_\bC$.

First, we inductively define the ``natural'' span-based conditon $\hat c$ for an arbitrary arrow-based condition $c$. Let $R\in\bC$.
\begin{itemize}
\item For an arbitrary $c=(R,p_1\ccdots p_w)\in \AC R$, let $\hat c=(R,\hat p_1\ccdots \hat p_w)$; 
\item For an arbitrary $p=(r,c)\in \AB R$, let $\hat p=(\id_R,r,\hat c)$.
\end{itemize}
%
It follows (by induction) that $\hat c\in \SC R$ for all $c\in \AC R$ and $\hat p\in \SB R$ for all $p\in \AB R$. In particular, an arrow-based branch $p=(r,c)$ (with root $R$) is translated to a span-based branch in which $R$ is the interface and the span consists of up-arrow $\id_R$ and down-arrow $r$.

The structure of morphisms remains essentially identical. Let $b,c\in \ABC$.
%
\begin{itemize}
\item For arbitrary $m:b\func c$ with $m=(t,o,m_1\ccdots m_w)$, let $\hat m=(t,o,\hat m_1\ccdots \hat m_w)$.
\end{itemize}
%
\begin{proposition}
$m$ is an ab-morphism from $b$ to $c$ if and only if $\hat m$ is an sb-morphism from $\hat b$ to $\hat c$. Moreover, $\hat \id_c=\id_{\hat c}$ for all ab-conditions $c$ and $\widehat{m;n}=\hat m;\hat n$ for all ab-morphisms $m,n$.
\end{proposition}
%
Now let $\cF$ map every $c\in \ABC$ to $\hat c$, and every $m\in \ABC(b,c)$ to $\hat m$.

\begin{theorem}
$\cF$ is a full and faithful functor from $\ABC$ to $\SBC$ such that, for all $c\in \ABC$ and all arrows $g$, $g\sat c$ if and only if $g\sat \cF(c)$.
\end{theorem}




