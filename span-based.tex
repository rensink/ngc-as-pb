\section{Span-based conditions}
\slabel{sb-conditions}

The peculiar definition of morphisms of ab-conditions just introduced, with arrows changing direction at each layer in order to ensure the preservation of models, clashes with a structural property of arrow-based conditions, namely the fact that the pattern of each subcondition contains a copy of the parent pattern because a branch consists of an arrow of $\cat{C}$ from the latter to the former. Thus if a morphism maps, say, the pattern $P^b$ of (a subcondition) of $b$ to a pattern $P^c$ of $c$, at the next layer (the copy of) $P^c$ must be mapped backwards to (the copy) of $P^b$. The consequence is that morphisms fail to exist even in very simple situations. An example is given by \fcite{ab-conditions}, in which we have seen that $c_3\entails c_1$ but there is no morphism providing evidence for this.
  
Inspired by these observation, we present the second main contribution of this paper: an original definition of nested condition where the duplication of structure just described does not occur, because at each level we can specify only the \emph{additional} structure (and how it is connected). This is achieved by replacing the branch arrows $a^c_i$ of a condition $c$ by \emph{spans}.

% We now present the main idea of this paper, inspired by on the observation that arrow-based conditions over graphs contain a lot of duplication: the children of a node essentially contain all the structure of that (parent) node, and add some constraints on top of it. It turns out to be possible to avoid this duplication by only specifying the \emph{additional} structure (and how it is connected). This is achieved by replacing the branch arrows $a^c_i$ of a condition $c$ by \emph{spans}.\footnote{We discuss and contrast an alternative way to resolve the same issue, using \emph{cospans} rather than spans, later in this paper.}\todo{AC: Remove the footnote and add a sentence in the future work section.}

A span $s = (f: I \to A, g:I \to B)$ over \cat{C}, which we will denote $\spanof f g$, is a pair of arrows of \cat{C} having the same source. Objects $A$ and $B$ are the \emph{source} and \emph{target} of $s$, respectively, while $I$ is its \emph{interface}; we sometimes write $s:A\to B$. Spans can be composed: if the target of $\spanof u d$ is equal to the source of $\spanof v w$, then $\spanof u d \spcomp \spanof v w=\spanof{v';u}{d';w}$ where $\spanof {v'}{d'}$ is the pullback span of $(d,v)$. The result is defined, because \cat{C} has all pullbacks, but non-deterministic in general because pullbacks are defined up to iso. The standard solution would be to consider spans up to isomorphism of their interface (yielding category {\SpanC} having the same objects of \cat{C} and equivalence classes of spans as arrows), but to avoid further technicalities we will gloss over this throughout the paper.

%An arrow $a:A \to B$ can be seen as a span $\spanof{id_A}{a}$. This defines an embedding of standard arrows of \cat{C} into spans over \cat{C} that preserves composition. Composing an arrow with a span, as in $a \spcomp \spanof f g$, is intended to mean $\spanof{id}{a}  \spcomp \spanof f g$.\todo{delete if not used}

\medskip\noindent
Span-based conditions are inductively defined as follows:

\begin{definition}[span-based condition]\dlabel{sb-condition}
  For any object $R$ of $\bC$, $\SC R$ (the set of \emph{span-based conditions} over $R$) and $\SB R$ (the set of \emph{span-based branches} over $R$) are the smallest sets such that
  \begin{itemize}
  \item $c\in \SC R$ if $c=(R,p_1\ccdots p_w)$ is a pair with $p_i\in \SB R$ for all $1\leq i\leq w$;
  \item $p\in \SB R$ if $p=(\spanof u d,c)$ where $u: I\to R, d:I\to P$ form a span of arrows of $\bC$ and $c\in \SC P$.
  \end{itemize}
\end{definition}
%
In the span $\spanof u d$ of a branch $p=(\spanof u d,c)$, $u$ stands for the \emph{up-arrow} and $d$ for the \emph{down-arrow}. As before, we use $|c|=w$ to denote the width of a span-based condition $c$, $R^c$ to denote its root, and $p^c_i=(s^c_i,c_i)$ with span $s^c_i=\spanof{u^c_i}{d^c_i}$ its $i$-th branch. Finally, we use $I^c_i$ for the interface of span $s^c_i$ and $P^c_i$ ($=R^{c_i}$) for its target (hence $u^c_i:I^c_i\to R^c$ and $d^c_i:I^c_i\rightarrow P^c_i$). In all these cases, we may omit the superscript $c$ if it is clear from the context. Pictorially, $c$ can be visualised as in \fcite{sb-condition}.
%
\begin{figure}[t]
\centering
\subcaptionbox
  {Condition $c=(R,p_1\ccdots p_w)$, with $p_i=(\spanof{u_i}{d_i},c_i)$ for $1\leq i\leq w$
   \flabel{sb-condition}}
  [.45\textwidth]
  {\begin{tikzpicture}[on grid]
  \node (R) {$R$};
  \node (I1) [below left=of R] {$I_1$};
  \node (R1) [below=1.2 of I1] {$R_1$};
  \node (c1) [triangle,below=.15 of R1.center] {$c_1$};
  \node [below=.8 of Rc] {$\cdots$};
  \node (In) [below right=of R] {$I_n$};
  \node (Rn) [below=1.2 of In] {$R_n$};
  \node (cn) [triangle,below=.15 of Rn.center] {$c_n$};

  \path (I1) edge[->] node[above left] {$u_1$} (R)
        (I1) edge[->] node[left] {$d_1$} (R1)
        (In) edge[->] node[above right] {$u_n$} (R)
        (In) edge[->] node[right] {$d_n$} (Rn);
\end{tikzpicture}
}
\qquad
\subcaptionbox
  {$g\sat c$, with responsible branch $p_i=(\spanof{u_i}{d_i},c_i)$ and witness $h$ such that $\spanof{u_i}{d_i}\commutes (g,h)$
   \flabel{sb-satisfaction}}
  [.45\textwidth]
  {\begin{tikzpicture}
  \node (Rc) {$R_c$};
  \node (I1) [below left=.3 and .2 of Rc.center] {$\cdots$};
  \node (Ip) [below=of Rc] {$I_p$};
  \node (In) [below right=.3 and .2 of Rc.center] {$\cdots$};
  \node (Rp) [below=.7 of Ip] {$R_{c_p}$};
  \node (cp) [triangle,below=.15 of Rp.center] {$c_p$};
  \node (G) [right=2 of Rc] {$G$};

  \path (Rc) edge[->] node[above] {$g$} (G)
        (Ip) edge[->] node[left] {$u_p$} (Rc)
        (Ip) edge[->] node[left] {$d_p$} (Rp)
        (Rp) edge[->] node[pos=0.3,below right] (h) {$h$} (G)
        (h) edge[draw=none] node[sloped,allow upside down] {$\nsat$} (cp);
\end{tikzpicture}
}
\vspace*{-2mm}
\caption{Visualisations for span-based conditions}
\end{figure}
%
As we will see in \sref{categories}, we can reconstruct an arrow-based condition from a span-based one.

As examples of span-based conditions, \fcite{sb-conditions} shows counterparts for the three arrow-based conditions in \fcite{ab-conditions}: both $b_1$ and $b_1'$ are equivalent to $c_1$ (which is rather non-obvious in the case of $b_1'$), whereas $b_2$ is equivalent to $c_2$ and $b_3$ to $c_3$.

\begin{figure}[t]
\centering
\begin{tikzpicture}[on grid,node distance=.8 and .7]
  \node[graph] (00) {\onenode{x}};
  \node[left=0 of 00.west] {$b_1'$};
  \node[graph,below=of 00] (I-01) {\twonode{x}{x'}}; 
  \node[graph,below=of I-01] (01) {\oneedge{x}{b}{x'}}; 
  \node[graph,below left=of 01] (I-011) {\onenode{x}};
  \node[graph,below=of I-011] (011) {\oneedge{x}{a}{y}};
  \node[graph,below right=of 01] (I-012) {\onenode{x'}};
  \node[graph,below=of I-012] (012) {\oneedge{x'}{c}{z}};
  
  \path
    (I-01) edge[->] (00)
    (I-01) edge[->] (01)
	(I-011) edge[->] (01)
	(I-011) edge[->] (011)
    (I-012) edge[->] (01)
    (I-012) edge[->] (012);
  
  \node[graph,right=3 of 00] (10) {\onenode{x}};
  \node[left=0 of 10.west] {$b_1$};
  \node[graph,below=of 10] (I-11) {\onenode{x}}; 
  \node[graph,below=of I-11] (11) {\oneloopleft{x}{b}}; 
  \node[graph,below left=of 11] (I-111) {\onenode{x}};
  \node[graph,below=of I-111] (111) {\oneedge{x}{a}{y}};
  \node[graph,below right=of 11] (I-112) {\onenode{x}};
  \node[graph,below=of I-112] (112) {\oneedge{x}{c}{z}};
  
  \path
    (I-11) edge[->] (10)
    (I-11) edge[->] (11)
	(I-111) edge[->] (11)
	(I-111) edge[->] (111)
    (I-112) edge[->] (11)
    (I-112) edge[->] (112);
  
  \node[graph,right=3 of 10] (20) {\onenode{x}};
  \node[left=0 of 20.west] {$b_2$};
  \node[graph,below=of 20] (I-21) {\onenode{x}}; 
  \node[graph,below=of I-21] (21) {\oneedge{x}{b}{y}}; 
  \node[graph,below left=of 21] (I-211) {\onenode{y}};
  \node[graph,below=of I-211] (211) {\oneloop{y}{a}};
  \node[graph,below right=of 21] (I-212) {\onenode{y}};
  \node[graph,below=of I-212] (212) {\oneedge{y}{c}{z}};
  
  \path
    (I-21) edge[->] (20)
    (I-21) edge[->] (21)
	(I-211) edge[->] (21)
	(I-211) edge[->] (211)
    (I-212) edge[->] (21)
    (I-212) edge[->] (212);

  \node[graph,right=3 of 20] (30) {\onenode{x}};
  \node[left=0 of 30.west] {$b_3$};
  \node[graph,below left=of 30] (I-31) {\onenode{x}}; 
  \node[graph,below=of I-31] (31) {\oneloop{x}{a}}; 
  \node[graph,below right=of 30] (I-32) {\onenode{x}}; 
  \node[graph,below=of I-32] (32) {\oneedge{x}{b}{y}}; 
  \node[graph,below=of 32] (I-321) {\onenode{y}}; 
  \node[graph,below=of I-321] (321) {\spangraph{v}{c}{y}{c}{z}}; 
  \node[graph,below=of 321] (I-3211) {\twonode{v}{z}}; 
  \node[graph,below=of I-3211] (3211) {\onenode z}; 

  \path (I-31) edge[->] (30)
		(I-31) edge[->] (31)
        (I-32) edge[->] (30)
		(I-32) edge[->] (32)
		(I-321) edge[->] (32)
		(I-321) edge[->] (321)
		(I-3211) edge[->] (321) 
		(I-3211) edge[->] node[right] {\mapping{v&z}} (3211);

  \path
    ($(10)+(-.2,-4)$) edge[<-,bend left=50,dotted] node[below] {$m$} ($(00)+(0,-4)$)
    ($(10)+(.2,-4)$) edge[<-,bend right=50,dotted] node[below] {$m'$} ($(20)+(-.2,-4)$)
    ($(20)+(.2,-4)$) edge[<-,bend right=50,dotted] node[below] {$m''$} ($(30)+(-.2,-4)$);
	
  \path[morphism]
    (01) edge[medge,->] (11)
	(111) edge[medge,bend left=30,->] (011)
	(112) edge[medge,bend left=30,->] (012)
	;
	
  \path[morphism]
    (21) edge[medge,->] (11)
	(111) edge[medge,bend right=30,->] (211)
	(112) edge[medge,bend right=30,->] (212)
	;
	
  \path[morphism]
    (32) edge[medge,bend left=20,->] (21)
	(212) edge[medge,bend right=30,->] (321)
	;
\end{tikzpicture}

\vspace*{-5mm}
\caption{Span-based conditions corresponding to \fcite{ab-conditions}. The dotted arrows indicate forward-shift morphisms (see \excite{sb-morphisms}), with component mappings in brown.}
\flabel{sb-conditions}
\end{figure}

\medskip\noindent The careful reader might have observed that \dcite{sb-condition} is essentially \dcite{ab-condition} instantiated to the category {\SpanC}, thus wondering if we are just repeating the theory of ab-conditions for one specific instance. This is not the case, though, as the definitions that follow (satisfaction, morphisms, \ldots) are phrased using arrows and diagrams of \cat{C}, not of {\SpanC}.\footnote{Technically, this would not fall in our framework because {\SpanC} is not a presheaf topos even if \cat{C} is.}

We now present the modified notion of satisfaction for span-based conditions. First we recognise that the purpose of an arrow $a$ in an ab-condition is essentially to establish ``correct" model/witness pairs $(g,h)$ --- namely, those pairs that commute with $a$ in the sense of satisfying $g=a;h$. We may write $a\commutes (g,h)$ to express this commutation relation. This is what we now formalise for spans, as follows:
\[ \spanof u d \commutes (g,h) \text{ if } u;g=d;h \enspace. \]
Satisfaction of sb-conditions is then defined as follows:

\begin{definition}[satisfaction of span-based conditions]\dlabel{sb-satisfaction}
  Let $c$ be a span-based condition and $g:R\to G$ an arrow from $c$'s root to an object $G$. We say that \emph{$g$ satisfies $c$}, denoted $g\sat c$, if there is a branch $p_i$ and an arrow $h:R_i\to G$ such that
  \begin{enumerate*}
  \item $s_i \commutes (g, h)$, and
  \item $h\nsat c_i$.
  \end{enumerate*}
\end{definition}
%
Note that this is entirely analogous fo \dcite{ab-satisfaction}, especially if we would retrofit the notation $a\commutes (g,h)$ there. Like before, we call $p_i$ the \emph{responsible branch} and $h$ the \emph{witness} of $g\sat c$. Satisfaction of sb-conditions is visualised in \fcite{sb-satisfaction}.

\subsection{Span-based root shifting}
\slabel{span-based}

To define morphisms of sb-conditions, we will go over the same ground as for ab-conditions, but take a slightly more abstract view --- as we did for satisfaction by introducing $\commutes$. Having recognised that the \emph{semantics} of a span in sb-conditions is essentially given by the set of model/witness pairs that it commutes with, we can in general consider relations over spans that either preserve or reflect this semantics, in the sense that if $s_1,s_2$ are related, then either $s_1\commutes (g,h)$ implies $s_2\commutes (g,h)$ (preservation) or $s_2\commutes (g,h)$ implies $s_1\commutes (g,h)$ (reflection). Typically, however, the model/witness pairs are not preserved or reflected precisely, but modulo some arrow that gets added to or erased from the model or witness.
%
\begin{itemize}
\item $(s_1,s_2)$ preserves models adding $v$ if $s_1\commutes (g,h)$ implies $s_2\commutes (v;g,h)$;
\item $(s_1,s_2)$ preserves models erasing $v$ if $s_1\commutes (v;g,h)$ implies $s_2\commutes (g,h)$;
\item $(s_1,s_2)$ reflects models adding $v$ if $s_2\commutes (g,h)$ implies $s_1\commutes (v;g,h)$;
\item $(s_1,s_2)$ reflects models erasing $v$ if $s_2\commutes (v;g,h)$ implies $s_1\commutes (g,h)$.
\end{itemize}
%
The above is defined on individual pairs. For a relation $\cR$ over spans, we say that $\cR$ preserves [reflects] models adding [erasing] $v$ if it consists of pairs that do so.

\begin{definition}[span source shifters]\dlabel{sb-shifters}
Let $X,Y$ be objects.
\begin{itemize}[topsep=\smallskipamount]
\item A \emph{span source shifter} $\cS$ from $X$ to $Y$ is a function from $X$-sourced spans to $Y$-sourced spans having the same target, say $Z$, such that for all spans $s:X\to Y,t:Y\to Z$:
%
\begin{equation}\eqlabel{span shifters are functors}
\cS(s \spcomp t) = \cS(s)\spcomp t \enspace.
\end{equation}
%
\item Now let $v=A\to B$ be an arrow. $\cS$ is a \emph{forward span source shifter for $v$} if it is a span source shifter from $A$ to $B$ that preserves models erasing $v$, i.e., such that for all spans $s:A\to C$ and all and all pairs $g:A\to G,h:C \to G$:
%
\begin{equation}\eqlabel{sb-forward condition}
s\commutes (v;g,h) \enspace\text{ implies }\enspace \cS(s) \commutes (g,h) \enspace.
\end{equation}
%
A forward span source shifter is called \emph{complete} if it also reflects models adding $v$; i.e., if the ``implies" in \eqref{sb-forward condition} is an ``if and only if".

\item Instead, $\cS$ is a \emph{backward span source shifter for $v$} if it is a span source shifter from $B$ to $A$ that reflects models erasing $v$, i.e., such that for all spans $s:B\to C$ and all pairs $g:B\to G,h:C\to G$:
%
\begin{equation}\eqlabel{sb-backward condition}
\cS(s) \commutes (v;g,h) \enspace\text{ implies }\enspace s\commutes (g,h) \enspace.
\end{equation}
%
A backward span source shifter is called \emph{complete} if it also preserves models adding $v$; i.e., if the ``implies" in \eqref{sb-backward condition} is an ``if and only if".
\end{itemize}
\end{definition}
%
Since span source shifters from $X$ are, just as in the arrow-based case, completely determined by $\cS(\id_X)$ (due to \eqref{span shifters are functors}), we can enumerate the choices of forward and backward shifters for $v$ that are generally available (without having additional information about $v$).
%
\begin{description}
\item[Forward shifting.] For $\cS$ to be a forward span source shifter for $v:A\to B$, there must be a span $t=\cS(\id_A):B\to A$, which automatically determines $\cS(s)=t\spcomp s$ for all $s:A\to C$. There are essentially two options: $t=\spanof{v}{\id_B}$ or $t=\spanof{\id_A}{x}$ for some section $x$ of $v$, i.e., such that $x;v=\id_B$.
% We denote the resulting span source shifters by $\cF_v^\circ$ and $\cF_v^x$, respectively.

\item[Backward shifting.] For $\cS$ to be a backward span source shifter for $v:A\to B$, there must be a span $t=\cS(\id_B):A\to B$, which automatically determines $\cS(s)=t \spcomp s$ for all $s:B\to C$. Again, there are essentially two options: $t=\spanof{\id_A}{v}$, or $t=\spanof{x}{\id_B}$ for some section $x$ of $v$, i.e., such that $x;v=\id_A$.
% We denote the resulting span source shifters by $\cB_v^\circ$ and $\cB_v^x$, respectively.
\end{description}
%
This gives rise to the following four candidate shifters for $v:A\to B$ (which still have to be shown to satisfy the conditions of \dcite{sb-shifters}):
%
\begin{align*}
\dF v & : s \mapsto \spanof{v}{\id_B}\spcomp s \\
\rF v x& : s \mapsto \spanof{\id_A}{x}\spcomp s \text{ where $x;v=\id_B$} \\
\dB v& : s \mapsto \spanof{\id_A}{v}\spcomp s \\
\rB v x& : s \mapsto \spanof{x}{\id_B}\spcomp s \text{ where $x;v=\id_A$}
\end{align*}
%
These are visualised in \fcite{sb-source shifters} (in a simplified construction that omits the intermediate steps of the span composition where possible). We call $\dF v$ and $\dB v$ \emph{direct} and $\rF v x$ and $\rB v x$ \emph{split}. It turns out that, indeed, each of these four options gives rise to a valid shifter, and the direct ones are complete.
%
\begin{figure}[t]
\centering
\begin{tikzpicture}[on grid]
  \node (F) {$\cF_v^\circ$};
  \node (A) [below left=.7 and .75 of F] {$A$};
  \node (B) [right=1.5 of A] {$B$};
  \node (I) [below=1.5 of A] {};
  \node (C) [below=1.5 of I] {$C$};

  \path (A) edge[->] node[above] {$v$} (B)
        (I) edge[->] node[left] {$u$} (A) 
        (I) edge[->] node[left] {$d$} node[right] {$d'$} (C)
		(I) edge[->] node [right] {$u'$} (B);
\end{tikzpicture}
\quad
\begin{tikzpicture}[on grid]
  \node (F) {$\cF_v^x$};
  \node (A) [below left=.7 and .75 of F] {$A$};
  \node (B) [right=1.5 of A] {$B$};
  \node (I) [below=1.5 of A] {};
  \node (C) [below=1.5 of I] {$C$};
  \node (B2) [below=1 of B] {$B$};
  \node (I2) [below=1.5 of B2] {};
  
  \path (A) edge[->] node[above] {$v$} (B)
        (I) edge[->] node[left] {$u$} (A) 
        (I) edge[->] node[left] {$d$} (C)
		(B2) edge[->] node[right] {$\id_B$} (B)
		(B2) edge[->] node[below] {$x$} (A)
		(I2) edge[->] node[right] {$u'$} (B2)
		(I2) edge[->] node[below] {$x'$} (I)
		(I2) edge[->] node[below] {$d'$} (C);

  \path	(I2) edge[-{Straight Barb[black,length=5pt,width=10pt]},white] +(-3mm,6mm);
\end{tikzpicture}
\quad
\begin{tikzpicture}[on grid]
  \node (F) {$\cB_v^\circ$};
  \node (A) [below left=.7 and .75 of F] {$A$};
  \node (B) [right=1.5 of A] {$B$};
  \node (I) [below=1.5 of B] {};
  \node (C) [below=1.5 of I] {$C$};
  \node (I2) [below=1.5 of A] {};
  
  \path (A) edge[->>] node[above] {$v$} (B)
        (I) edge[->] node[right] {$u$} (B) 
        (I) edge[->] node[right] {$d$} (C)
		(I2) edge[->>] node[above] {$v'$} (I)
		(I2) edge[->] node[left] {$u'$} (A)
		(I2) edge[->] node[below left] {$d'$} (C);

  \path	(I2) edge[-{Straight Barb[black,length=5pt,width=10pt]},white] +(+3mm,3mm);
\end{tikzpicture}
\quad
\begin{tikzpicture}[on grid]
  \node (F) {$\cB_v^x$};
  \node (A) [below left=.7 and .75 of F] {$A$};
  \node (B) [right=1.5 of A] {$B$};
  \node (B2) [below=of B] {$B$};
  \node (I) [below=of B2] {};
  \node (C) [below=of I] {$C$};
  
  \path (A) edge[->] node[above] {$v$} (B)
        (I) edge[->] node[right] {$u$} (B2) 
        (I) edge[->] node[right] {$d$} node[left] {$d'$} (C)
		(B2) edge[->] node[right] {$\id_B$} (B)
		(B2) edge[->] node[below] {$x$} (A)
		(I) edge[->,bend left] node[below left] {$u'$} (A);
\end{tikzpicture}

\caption{Forward and backward span source shifters. In each case, $\spanof{u'}{d'}$ is the image of $\spanof{u}{d}$.}
\flabel{sb-source shifters}
\end{figure}
%
\begin{proposition}[span source shifters]\plabel{sb-source shifters}
\begin{enumerate}[topsep=\smallskipamount]
\item $\cF_v^\circ$ is a complete forward span source shifter for $v$, and $\cF_v^x$ is a forward span shifter for $v$ if $v$ is a split epi with section $x$.
\item $\cB_v^\circ$ is a complete backward span source shifter for $v$ if $v$ is epi, and $\cB_v^x$ is a backward span shifter for $v$ if $v$ is a split epi with section $x$.
\end{enumerate}
\end{proposition}
%
\iffest
The proof (in \acite{proofs}) can be constructed directly from the diagrams in \fcite{sb-source shifters}.
\fi
\begin{fullorname}
\begin{proof}
In the proof, we make use of the diagrams (especially the arrow names) in \fcite{sb-source shifters}.
\begin{enumerate}
\item We first show that $\cF_v^\circ$ is a complete forward span source shifter. Let $g:B\to G$ and $h:C\to G$; then $\spanof{u}{d}\commutes (g,h)$ and $\spanof{u'}{d'}\commutes (g,h)$ are both equivalent to $u;v;g=d;h$, hence they are equivalent to one another. It follows that $\cF_v^\circ$ both  preserves models erasing $v$ and reflects models adding $v$.

We now show that $\cF_v^x$ is a forward source shifter, i.e., that it preserves models erasing $v$. Let $g:B\to G$ and $h:C\to G$ be such that $\spanof{u}{d}\commutes (v;g,h)$, meaning $u;v;g=d;h$. It follows that $x';u;v;g=x';d;h$ and hence $u';g=u';x;v;g=d';h$, hence $\spanof{u'}{d'}\commutes (g;h)$.

\item We first show that $\cB_v^\circ$ is a complete backward span source shifter for $v$ if $v$ is epi. First note that in toposes the pullback of an epi is epi; hence, as already indicated in the figure, $v'$ is epi. Now let $g:B\to G$ and $h:C\to G$; then $\spanof{u}{d}\commutes (g,h)$ comes down to $u;g=d;h$ whereas $\spanof{u'}{d'}\commutes (v;g,h)$ comes down to $u';v;g=d';h$, which is equivalent to $v';u;g=v';d;g$. Because $v'$ is epi, $u;g=d;h$ if and only if $v';u;g=v';d;g$, implying that $\cB_v^\circ$ both reflects models erasing $v$ and preserves models adding $v$.

We now show that $\cB_v^x$ is a backward span source shifter for $v$, i.e., that it reflects models erasing $v$. Let $g:B\to G$ and $h:C\to G$ be such that $\spanof{u'}{d'}\commutes (v;g,h)$, meaning $u';v;g=d';h$. It follows that $u;g=u;x;v;g=u';v;g=d';h=d;h$, hence $\spanof u d\commutes (g,h)$.\qed
\end{enumerate}
\end{proof}
\end{fullorname}
%
%The ability to forward-shift and backward-shift in many different ways means that there are equally many options for building morphisms. In fact, however, we will show later that all above root shifters for $v$ (both forward and backward) are subsumed by $\cF_v^\circ$, in the sense that entailment or equivalence of sb-conditions ``explained" by morphisms based on any of the (other) known root shifters can also be ``explained" by morphisms based on $\cF_v^\circ$.

\medskip\noindent
The step from span source shifters to span root shifters is completely identical to the arrow-based case; we omit the definition here. A new aspect is the following.

\begin{proposition}
The trivial root shifter $\cI_{A,B}$ is both a complete forward span root shifter and a complete backward span root shifter for any arrow $v:A\to B$. 
\end{proposition}
%
Span root shifters satisfy the analogous properties of arrow root shifters formulated in Propositions \pref{ab-root shifters preserve} (model preservation) and \pref{ab-shifters compose} (composition), both extended with the case of complete shifters. 
\iffull
As the proofs are essentially identical, we omit them here. 
\fi

\begin{proposition}[span root shifters preserve models]\plabel{sb-shifters preserve}
Let $v:A\to B$.
\begin{enumerate}[topsep=\smallskipamount]
\item If $c\in \SC A$ and $\cF$ is a forward span root shifter for $v$ that is defined on $c$, then $v;g\sat c$ implies $g\sat \cF(c)$. If, moreover, $\cF$ is complete then also $g\sat \cF(c)$ implies $v;g\sat c$.
\item If $c\in \SC B$ and $\cB$ is a backward span root shifter for $v$ that is defined on $c$, then $v;g\sat \cB(c)$ implies $g\sat c$. If, moreover, $\cB$ is complete then also $v;g\sat \cB(c)$ implies $g\sat c$.
\end{enumerate}
\end{proposition}
%
Forward and backward shifters again compose, also for complete ones.
\iffest
The precise statement is given in \pcite{sb-shifters compose}.
\fi

\begin{fullorname}[\pname{sb-shifters compose}]
\begin{proposition}[span shifters compose]\plabel{sb-shifters compose}
\begin{enumerate}[topsep=\smallskipamount]
\item If $\cU$ is a span source [root] shifter from $X$ to $Y$ and $\cV$ a span source [root] shifter from $Y$ to $Z$, then $\cU;\cV$ (understood as partial function composition) is a span source [root] shifter from $X$ to $Z$. 
\item If $\cU$ and $\cV$ are [complete] forward span source [root] shifters for $u: X\to Y$ and $v: Y \to Z$, respectively, then $\cU;\cV$ is a [complete] forward source [root] shifter for $u;v$.
\item Dually, if $\cU$ and $\cV$ are [complete] backward span source [root] shifters for $u: Y\to X$ and $v: Z \to Y$, respectively, then $\cU;\cV$ is a [complete] backward span source [root] shifter for $v;u$.
\end{enumerate}
\end{proposition}
\end{fullorname}
%
\begin{comment}
\medskip\noindent
In fact we also need the following auxiliary composition property for direct span shifters.\todo{AR: Added AC: ok} This immediately follows from the associativity of span composition combined with the fact that spans with identity left-legs and identity right-legs compose; we omit the proof.

\begin{lemma}[direct span shifters compose]\llabel{direct sb-shifters compose}
Let $t:A\to B,v:B\to C$ be arrows; then
\begin{enumerate*}[topsep=\smallskipamount]
\item $\dF t;\dF v=\dF{t;v}$, and 
\item $\dB v;\dB t=\dB{t;v}$.
\end{enumerate*}
\end{lemma}
\end{comment}
%
We call the span source and root shifters of the shape $\dF v$, $\rF v x$, $\dB v$, $\rB v x$ and $\cI_{A,B}$ \emph{elementary}. It is not the case that every span shifter is elementary; e.g., $\dF{v_1};\rF{v_2}{x_2}$ is in general not equal to $\dF{v_1;v_2}$ or to $\rF{v_1;v_2} x$ for any $x$. However, for the remainder of this paper we will restrict ourselves to span shifters that are at least \emph{composed} from elementary ones, in the following sense:

\begin{assumption}\asslabel{sb-shifters} Let $\cS$ be a span root shifter for $v$.
\begin{itemize}[topsep=\smallskipamount]
\item If $\cS$ is a forward shifter, then $\cS = \cS_1;\cdots;\cS_n$ where each $\cS_i$ is an elementary forward shifter for some $v_i$ such that $v=v_1;\cdots;v_n$. If, moreover, $\cS$ is complete then all $\cS_i$ are complete --- in which case either $\cS$ is trivial or $\cS_i=\dF{v_i}$ for all $i$ and $\cS=\dF v$.
\item If $\cS$ is a backward shifter, then $\cS = \cS_1;\cdots;\cS_n$ where each $\cS_i$ is an elementary backward shifter for some $v_i$ such that $v=v_n;\cdots;v_1$. If, moreover, $\cS$ is complete then all $\cS_i$ are complete --- in which case either $\cS$ is trivial or $\cS_i=\dB{v_i}$ for all $i$ and $\cS=\dB v$.
\end{itemize}
\end{assumption}
%
\subsection{Span-based morphisms}

A final change with respect to the arrow-based case is that we also abstract the commutation condition in \dcite{ab-morphism} by relying instead on a family of \emph{pattern shift relations} --- which in a sense complement the role of root shifters.
%
\begin{definition}[pattern shift]
Let $v:P_1\to P_2$ be an arrow. \emph{Pattern shift for $v$} is the relation $\cP_v$ over spans including all pairs $(s_1,s_2)$ such that  $s_i=\spanof{u_i:I_i\to A}{d_i:I_i\to P_i}$ for $i=1,2$ (for some $A$), and there is an arrow $k:I_1\to I_2$ for which both halves of the following diagram commute:
\begin{equation}\eqlabel{pattern shift}
\begin{tikzpicture}[on grid,baseline=(I1.center)]
\node (A) {$A$};
\node (I1) [below left=1 and 1.5 of A] {$I_1$};
\node (P1) [below=1.5 of I1] {$P_1$};
\node (I2) [below right=1 and 1.5 of A] {$I_2$};
\node (P2) [below=1.5 of I2] {$P_2$};
\path
  (I1) edge[->] node[above left,inner sep=1] {$u_1$} (A)
  (I1) edge[->] node[left] {$d_1$} (P1)
  (I2) edge[->] node[above right,inner sep=1] {$u_2$} (A)
  (I2) edge[->] node[right] {$d_2$} (P2);
\path[morphism]
  (P1) edge[->] node[above] {$v$} (P2);
\path[red,color=red]
  (I1) edge[->] node[below] {$k$} (I2);
\end{tikzpicture}
\end{equation}
\emph{Conservative pattern shift for $v$} is the subrelation $\ccP_v\subseteq \cP_v$ consisting of span pairs for which, in addition, the lower square of \eqref{pattern shift} is a pushout.
\end{definition}
%
The following lists some essential properties of pattern shifting.

\begin{proposition}[pattern shift properties]\plabel{pattern shift}
Let $v:P_1\to P_2$ be an arrow, and let $(s_1,s_2)\in \cP_v$.
\begin{enumerate}[topsep=\smallskipamount]
\item\label{pattern-preserves} For all arrows $g,h$, if $s_2\commutes (g,h)$ then $s_1\commutes (g,v;h)$;
\item\label{pattern-transitive} For all arrows $t:P_2\to P_2$, if $(s_2,s_3)\in \cP_t$ than  $(s_1,s_3)\in \cP_{v;t}$;
\item\label{pattern-congruence} For all spans $s$, if $s\spcomp s_1$ and $s \spcomp s_2$ are defined then $(s\spcomp s_1,s \spcomp s_2)\in \cP_v$.
\end{enumerate}
If $(s_1,s_2)\in \ccP_v$, then in addition:
\begin{enumerate}[resume,topsep=\smallskipamount]
\item\label{conservative-reflects} For all arrows $g,h$, if $s_1\commutes (g,h)$ then $h=v;h'$ and $s_2\commutes (g,h')$;
\item\label{conservative-transitive} For all arrows $t:P_2\to P_2$, if $(s_2,s_3)\in \ccP_t$ then $(s_1,s_3)\in \ccP_{v;t}$;
\item\label{conservative-congruence} For all spans $s=\spanof{a}{\id}$, if $s\spcomp s_1$ and $s\spcomp s_2$ are defined then $(s\spcomp s_1,s\spcomp s_2)\in \ccP_v$.
\end{enumerate}
\end{proposition}
%
\begin{fullorname}
\begin{proof}
Let $v:P_1\to P_2$ and $t:P_2\to P_3$ be arrows, and let $(s_1,s_2)\in \cP_v$. Let $s_i=\spanof{u_i:I_i\to A}{d_i:I_i\to P_i}$ for $i=1,2,3$.
\begin{enumerate}[topsep=\smallskipamount]
\item Assume $s_2\commutes (g,h)$, hence $u_2;g=d_2;h$. It follows that $u_1;g=k;u_2;g=k;d_2;h=d_1;v;h$, hence $s_1\commutes (g,v;h)$.

\item Assume $(s_2,s_3)\in \cP_{t}$; hence there are $k_v$ and $k_t$ making the pentagons of \eqcite{pattern shift} commute for $v$ and $t$, respectively. Then $k_v;k_t$ likewise satisfies the role of $k$ in \eqref{pattern shift} for $v;t$. If, moreover, the lower square of the constituent pentagons (for $v$ and $t$) are pushouts, then by pushout composition so is the lower square of the composed pentagon, which establishes clause \ref{conservative-transitive}.

\item Let $s=\spanof{b:J\to B}{a:J\to A}$. The following figure shows the pentagon obtained for $(s \spcomp s_1,s \spcomp s_2)$.
\begin{center}
\begin{tikzpicture}[on grid]
\node (A) {$A$};
\node (J1) [left=1.5 of A] {};
\node (J2) [right=1.5 of A] {};
\node (I1) [below left=1 and 1.5 of A] {$I_1$};
\node (P1) [below=of I1] {$P_1$};
\node (I2) [below right=1 and 1.5 of R] {$I_2$};
\node (P2) [below=of I2] {$P_2$};
\path
  (I1) edge[->] node[above left,near end,inner sep=1] {$u_1$} (A)
  (I1) edge[->] node[left] {$d_1$} (P1)
  (I2) edge[->] node[above right,near end,inner sep=1] {$u_2$} (A)
  (I2) edge[->] node[right] {$d_2$} (P2);
\path[morphism]
  (P1) edge[->] node[above] {$v$} (P2);
\path
  (I1) edge[->] node[above] {$k$} (I2);

\node (J) [above=1.5 of A] {$J$};
\node (B) [above=of J] {$B$};
\node (J1) [below left=1 and 1.5 of J,inner sep=1pt] {$J_1$};
\node (J2) [below right=1 and 1.5 of J,inner sep=1pt] {$J_2$};
\path (J) edge[->] node[right] {$b$} (B)
      (J) edge[->] node[right] {$a$} (A)
	  (J1) edge[->] node[above left,inner sep=1] {$u'_1$} (J)
	  (J2) edge[->] node[right] {$a_2$} (I2)
	  (J2) edge[->] node[above right,inner sep=1] {$u'_2$} (J)
	  (J1) edge[->] node[left] {$a_1$} (I1);
\path (J1) edge[-{Straight Barb[black,length=6pt,width=17pt]},white] +(5mm,-2.5mm);
\path (J2) edge[-{Straight Barb[black,length=6pt,width=17pt]},white] +(-5mm,-2.5mm);

\path[red,color=red]
  (J1) edge[->] node[near start,below] {$k'$} (J2);
\end{tikzpicture}
\end{center}
%
Given that $J_2$ is the pullback object of $(a,u_2)$ and $u'_1;a=a_1;u_1=a_1;k;u_2$, the red $k'$ uniquely exists such that $k';u'_2=u'_1$ and $k';a_2=a_1;k$. Hence $k'$ satisfies the properties of the required mediating morphism between $s\spcomp s_1=\spanof{u'_1;b}{a_1;d_1}$ and $s\spcomp s_2=\spanof{u'_2;b}{a_2;d_2}$.
\end{enumerate}
%
Now let $(s_1,s_2)\in \ccP_v$.
%
\begin{enumerate}[resume,topsep=\smallskipamount]
\item Assume $s_1\commutes (g,h)$ for $g:A\to G,h:P_1\to G$, hence $u_1;g=d_1;h$. It follows that $k;u_2;g=d_1;h$, hence by the pushout property of the lower square of \eqref{pattern shift} there is a unique $e:P_2\to G$ such that $u_2;g=d_2;e$ and $h=v;e$, implying $s_2\commutes (g,e)$.

\item Shown as part of the proof of clause \ref{pattern-transitive} above.

\item In the case where $s=\spanof{a}{\id_A}$ for some arrow $a$, the composition $s \spcomp s_i$ is simply $\spanof{a;u_i}{d_i}$. The lower square of the pentagon \eqref{pattern shift} for $(s\spcomp s_1,s\spcomp s_2)$ is then the same as that for $(s_1,s_2)$, hence it is a pushout.
\qed
\end{enumerate}
\end{proof}
\end{fullorname}
%
\iffull
Hence pattern shift preserves models (clause \ref{pattern-preserves}) and conservative pattern shift reflects models (clause \ref{conservative-reflects}), both ``modulo" the arrow $v$; both pattern shift and conservative pattern shift are transitively closed (modulo the arrow: clauses \ref{pattern-transitive} and \ref{conservative-transitive}); finally, pattern shift is a congruence with respect to prefixing an arbitrary span (clause \ref{pattern-congruence}), while conservative pattern shift is a congruence with respect to prefixing a span having an iso as right leg  (clause \ref{conservative-congruence}).

\fi
For the analogy with the arrow-based case, let $\cA_v=\setof{(a_1,a_2)\mid a_2=a_1;v}$; this constitutes a set of ``arrow pattern shift pairs" that play exactly the role captured by clause \ref{pattern-preserves}, but with respect to the satisfaction relation of ab-conditions. The conservative case (which we did not go into for ab-morphisms) would correspond to $\ccA_v=\setof{(a_1,a_2)\in \cA\mid a_1=a_2;x \text{ with } x;v=\id}$.

Using pattern shift, we define morphisms over sb-conditions as follows.
%
\begin{definition}[span-based (complete) condition morphism]\dlabel{sb-morphism}
  Given two sb-conditions $b,c$, a \emph{forward-shift [backward-shift] sb-morphism} $m:b\to c$ is a pair $(o,(v_1,m_1)\ccdots (v_{|b|},m_{|b|}))$ where
  \begin{itemize}[topsep=\smallskipamount]
  \item $o:[1,|b|]\to [1,|c|]$ is a function from $b$'s branches to $c$'s branches;
  \item for all $1\leq i\leq |b|$, $v_i:P^c_{o(i)}\to P^b_i$ is a arrow from the pattern of $p^c_{o(i)}$ to that of $p^b_i$ such that $(s^c_{o(i)},s^b_i)\in \cP_{v_i}$;
  \item \emph{Forward shift:} for all $1\leq i\leq |b|$, there is a forward root shifter $\cF_i$ for $v_i$ such that $m_i:\cF_i(c_{o(i)})\to b_i$ is a forward-shift morphism;
  \item \emph{Backward shift:} for all $1\leq i\leq |b|$, there is a backward root shifter $\cB_i$ for $v_i$ such that $m_i:c_{o(i)}\to \cB(b_i)$ is a backward-shift morphism.
  \end{itemize}
  The morphism $m$ is called \emph{direct} if all $\cF_i$ [$\cB_i$] and $m_i$ are direct, and is called \emph{complete} if $o$ is surjective and for all $1\leq i\leq |b|$, $(s^c_{o(i)},s^b_i)\in \ccP_{v_i}$, $\cF_i$ [$\cB_i$] is complete and $m_i$ is complete.
\end{definition}
%
Note that, by \asscite{sb-shifters}, only direct (forward or backward) span root shifters are complete, hence also only direct sb-morphisms can be complete.

Just as for the arrow-based case, one of the most essential properties of span-based condition morphisms is that they preserve models. In addition, complete span-based condition morphisms also reflect models. The following lifts \pcite{ab-morphisms preserve models} to sb-conditions, and extends it with the case for completeness.

\begin{proposition}[sb-condition morphisms preserve models]
                   \plabel{sb-morphisms preserve models}
Let $b,c \in \SC{R}$ be span-based conditions. If $m:b\to c$ is an sb-morphism, then $g\sat b$ implies $g \sat c$ for all arrows $g:R\to G$. Moreover, if $m$ is complete, then also $g\sat c$ implies $g\sat b$ for all $g$.
\end{proposition}
%
\begin{fullorname}
\begin{proof}
By induction on the depth of $m$. If $m:b\to c$ is an sb-morphism and $g\sat b$, then let $p^b_i$ be the responsible branch and $h$ the witness such that $\spanof{u^b_i}{d^b_i}\commutes (g,h)$ and $h\nsat b_i$, and let $j=o(i)$. Since $(s^c_j,s^b_i)\in \cP_{v_i}$, it follows by \pcite{pattern shift}.\ref{pattern-preserves} that $\spanof{u^c_j}{d^c_j}\commutes (g,v_i;h)$. Now assume (ad absurdum) that $v_i;h\sat c_j$. This will lead to a contradiction, hence $g\sat c$ with responsible branch $p^c_j$ and witness $v_i;h$.
\begin{itemize}[topsep=\smallskipamount]
\item For the forward-shift case, due to \pref{sb-shifters preserve} it follows that $h\sat \cF_i(c_j)$, and hence by the induction hypothesis the existence of $m_i:\cF_i(c_j)\to b_i$ implies $h\sat b_i$, contradicting the above.

\item For the backward-shift case, by the induction hypothesis the existence of $m_i:c_j\to \cB_i(b_i)$ implies $v_i;h\sat \cB_i(b_i)$, and hence due to \pref{sb-shifters preserve} it follows that $h\sat b_i$, contradicting the above.
\end{itemize}
If $m$ is complete and $g\sat c$, then let $p^c_j$ be the responsible branch and $h$ the witness such that $\spanof{u^c_j}{d^c_j}\commutes (g,h)$ and $h\nsat c_j$. Let $i$ be such that $j=o(i)$ (which exists because $o$ is surjective). Since $(s^c_j,s^b_i)\in \ccP_{v_i}$, it follows by \pcite{pattern shift}.\ref{conservative-reflects} that $\spanof{u^b_i}{d^b_i}\commutes (g,h')$ for some $h'$ such that $h=v_i;h'$. Now assume (ad absurdum) that $h'\sat b_i$. This will lead to a contradiction, hence $g\sat b$ with responsible branch $p^b_i$ and witness $h'$.
\begin{itemize}[topsep=\smallskipamount]
\item For the forward-shift case: by the induction hypothesis, the existence of the complete $m_i:\cF_i(c_j)\to b_i$ implies $h'\sat \cF_i(c_j)$. Since $\cF_i$ is complete, due to \pref{sb-shifters preserve} it follows that $h=v_i;h'\sat c_j$, contradicting the above.
\item For the backward-shift case, since $\cB_i$ is complete, due to \pref{sb-shifters preserve} it follows that $h=v_i;h'\sat \cB_i(b_i)$, and hence by the induction hypothesis the existence of the complete $m_i:c_j\to \cB_i(b_i)$ implies $h\sat c_j$, contradicting the above.
\qed
\end{itemize}
\end{proof}
\end{fullorname}
%
\begin{example}\exlabel{sb-morphisms}
For the span-based conditions in \fcite{sb-conditions}, we have a complete forward-shift morphism $m:b_1\to b_1'$, providing evidence that $b_1\equiv b_1'$; moreover, there are also morphisms $m':b_1\to b_2$ and $m'':b_2\to b_3$. All the required morphism arrows are already shown in the figure. Note that, in \excite{ab-morphisms}, we observed that there does \emph{not} exist an arrow-based morphism from $c_1$ to $c_2$; hence the existence of a span-based morphism between the equivalent $b_1$ and $b_2$ is concrete evidence for our claim, in \scite{categories} below, that the structure of span-based conditions allows more morhpisms between them.
\end{example}
%
We now once more return to the level of categories, using the exact same constructions for identity morphisms and morphism composition as in the arrow-based case (see \eqref{morphism composition} and \eqref{id-morphism}).
\iffull
To repeat, if $b,c,e$ are sb-conditions and $m\of b\to c,n\of c\to e$ are sb-morphisms with $|b|=w$:
%
\begin{align*}
\id_b & = (\id_{[1,w]},(\id_{P^b_1},\id_{b_1})\cdots
                       (\id_{P^b_w},\id_{b_w})) \\
m;n & = (o^m;o^n, (v^n_{o^m(1)};v^m_1,n_{o^m(1)};m_1)\cdots
                  (v^n_{o^m(w)};v^m_w,n_{o^m(w)};m_w)) \enspace.
\end{align*}
%
The next lemma lifts \lcite{ab-morphisms compose} to sb-morphisms and extends it with the notion of completeness.

\begin{lemma}[identities and composition of sb-morpisms]\llabel{sb-morphisms compose}
Let $b,c,e$ be span-based conditions and $m\of b\to c,n\of c\to e$ span-based condition morphisms.
\begin{enumerate}[topsep=\smallskipamount]
\item $\id_b$ is both a complete forward- and a complete backward-shift morphism from $b$ to $b$;
\item If $m$ and $n$ are [complete] forward-shift morphisms, then $m;n$ is a [complete] forward-shift morphism from $c$ to $e$;
\item If $m$ and $n$ are backward-shift morphisms, then $m;n$ is a backward-shift morphism from $c$ to $e$.
\end{enumerate}
\end{lemma}
\else
As in the case of ab-conditions (\lcite{ab-morphisms compose}), identities and morphism composition have the expected properties. In fact, also complete forward-shift morphisms compose, but (surprisingly, at least in our view) backward-shift complete morphisms do not.
\fi
We can therefore lift and extend \thcite{ab-categories} to the span-based case.

\begin{theorem}[categories of span-based conditions]\thlabel{sb-categories}
Consider the following categories having sb-conditions as objects and different arrows: 
$\SBC^\forw$ with forward-shift morphims, $\SBC^{\back}$ with backward-shift morphisms, $\SBC^{\forw{\circ}}$ with complete forward-shift morphisms, $\SBC^{\entails}$ with the preorder of semantic entailment, and $\SBC^\equiv$ with the preorder (actually, equivalence) of semantic equivalence. These are well-defined, and there are identity-on-objects functors $\SBC^\forw \to \SBC^{\entails}$, $\SBC^\back \to \SBC^{\entails}$ and $\SBC^{\forw{\circ}} \to \SBC^{\equiv}$.
\end{theorem}
%
We end our presentation of sb-condition morphisms with the functoriality of span root shifters, lifting \pcite{ab-shifters are functors} to the span-based case.

\begin{proposition}[span source shifters yield functors]\plabel{sb-shifters are functors}
For any pair of objects $X,Y$ and any span source shifter from $X$ to $Y$, the span root shifter $\bar\cS$ is a functor from $\SBC^\forw(X)$ to $\SBC^\forw(Y)$ and from $\SBC^\back(X)$ to $\SBC^\back(Y)$.
\end{proposition}

\subsection{From arrow-based to span-based and back}
\slabel{categories}

Span-based conditions are richer in morphisms than arrow-based conditions, but not richer in semantics. In this subsection we show the existence of faithful, semantics-preserving functors from the categories of forward- and backward-shift arrow-based conditions to their span-based counterparts, and we also show equivalence of the entailment-based categories.

First, we inductively define the ``natural'' span-based conditon $\cN(c)$ for an arbitrary arrow-based condition $c\in \AC R$ and $\cN(p)$ for a branch $p\in \AB R$.
\begin{align*}
\cN: c = (R,p_1\ccdots p_w) & \mapsto (R,\cN(p_1)\ccdots \cN(p_w)) \\
      p=(a,c) & \mapsto (\spanof{\id_R}{a},\cN(c)) \enspace.
\end{align*}
%
It follows (by induction) that $\cN(c)\in \SC R$ and $\cN(p)\in \SB R$. This transformation fully preserves the (satsfaction-based) semantics of conditions.

\begin{proposition}[$\cN$ preserves semantics]\plabel{N preserves semantics}
Let $c\in \AC R$. For all $g\of R\to G$, $g\sat c$ iff $g\sat\cN(c)$. As a consequence, for all $b\in \AC R$, $b\entails c$ if and only if $\cN(b)\entails \cN(c)$.
\end{proposition}
%
This is actually a consequence of \pcite{A preserves semantics} in combination with \lcite{N-A} below. Morphisms remain essentially identical under $\cN$:
%
\[ \cN: (o,(v_1,m_1)\ccdots (v_w,m_w)) \mapsto (o,(v_1,\cN(m_1))\ccdots (v_w,\cN(m_w))) \enspace. \]
%
$\cN$ both preserves and reflects morphisms, in the following sense.
%
\begin{proposition}
$m\of b\to c$ is a forward-shift [backward-shift] ab-condition morphism if and only if $\cN(m)\of \cN(b)\to \cN(c)$ is a forward-shift [backward-shift] sb-condition morphism.
\end{proposition}
%
\begin{proof}[sketch]
There are two things to be shown: namely,
\begin{enumerate*}[label=\emph{(\roman*)}]
\item the action of arrow-based forward and backward shifters on $a$ coincides with that of span-based forward and backward shifters on $\spanof{id}{a}$, and
\item the commutation condition $a_1=v;a_2$ (for ab-condition morphisms) holds if and only if $(\spanof{\id}{a_1},\spanof{\id}{a_2})\in \cP_v$.
\end{enumerate*}
The latter is immediate. For the former, briefly using $\hat \cS$ to denote span shifters, the essential property is that $\hat\cB_v^\circ(\spanof{\id}{a})=\spanof{id}{\cB_v(a)}$ and $\hat\cF_v^x(\spanof{\id}{a})=\spanof{id}{\cF_v^s(a)}$. (Span shifters of the form $\hat\cB_v^x$ or $\hat\cF_v^\circ$ do not preserve the property that the left span-leg is the identity, hence are not suitable to support morphisms from $\cN(b)$ to $\cN(c)$.)
\qed
\end{proof}
%
Since identities and morphism composition are obviously preserved by $\cN$, we have the following result, also illustrated in \fcite{categories}:

\begin{theorem}[arrow-based to span-based functors]\thlabel{ab-to-sb functors}
$\cN$ is a faithful functor from $\ABC^\forw$ to $\SBC^\forw$, from $\ABC^\back$ to $\SBC^\back$ and (with the appropriate, obvious mapping on arrows) from $\ABC^{\entails}$ to $\SBC^{\entails}$, which commute with the functors of \thcite{ab-categories} and \thcite{sb-categories}.
\end{theorem}
%
From span-based to arrow-based conditions with forward- or backward-shift morphisms, there does not exist a functor, as the span-based categories are richer in morphisms --- for instance, \fcite{sb-conditions} shows an example morphism $m':b_1\to b_2$, whereas no morphism exists between the equivalent $c_1$ and $c_2$ in \fcite{ab-conditions} (confer Examples \exref{ab-morphisms} and~\exref{sb-morphisms}). However, we now show that span-based conditions themselves are not richer in semantics than arrow-based ones: there does exist a semantics-preserving mapping from $\SC A$ to $\AC A$ for all $A$.

Let $(d'_s,u'_s)$ denote the pushout cospan for a span $s=\spanof u d$. We inductively define $\cA$ on sb-conditions and sb-branches, as follows:
%
\begin{align*}
\cA: (R,p_1\ccdots p_w) & \mapsto (R,\cA(p_1)\ccdots \cA(p_w)) \\
     (s,c) & \mapsto (d'_s,\cA(\dF{u'_s}(c))) \enspace.
\end{align*}
%
For instance, taking again the ab-conditions in \fcite{ab-conditions} and their sb-condition counterparts in \fcite{sb-conditions}, it can be checked that $\cA(b_1)=\cA(b'_1)=c_1$, $\cA(b_2)=c_2$ and $\cA(b_3)=c_3$.

$\cA$ acts as the (left) inverse of the action of $\cN$ on objects, due to $d'_s=\id$ and $u'_s=a$ for $s=\spanof{\id}{a}$:
%
\begin{lemma}\llabel{N-A}
For any arrow-based condition $c$, $\cA(\cN(c))=c$.
\end{lemma}
%
The following property is the counterpart of \pcite{N preserves semantics}, and actually implies it thanks to \lcite{N-A}:
%
\begin{proposition}[$\cA$ preserves semantics]\plabel{A preserves semantics}
If $c\in SC(R)$, then $\cA(c)$ is an arrow-based condition in $\AC R$ such that for all $b\of R\to G$, $g\sat c$ iff $g\sat\cA(c)$. In consequence, for all $b\in \SC R$, $b\entails c$ if and only if $\cA(b)\entails \cA(c)$.
\end{proposition}
%
\begin{proof}[sketch]
This is shown by induction on the depth of $c$, using that $g=d'_s;h$ iff $s\commutes (g,u'_s;h)$ for any model/witness pair $(g,h)$, in combination with the fact that $\dF{u'_s}$ preserves models adding $u'_s$ and reflects them erasing $u'_s$ (\pcite{sb-source shifters}).\qed
\end{proof}
%
This implies that (as claimed above) span-based conditions are semantically not more expressive than arrow-based categories. The formal statement comes down to an equivalence of the respective categories, also illustrated in \fcite{categories}.
%
\begin{theorem}[arrow-based to span-based functor]\thlabel{sb-to-ab functor}
$\cA$ is a (full and faithful) functor from $\SBC^{\entails}$ to $\ABC^{\entails}$, surjective on objects, which (together with its left and right adjoint $\cN$) establishes that the categories are equivalent.
\end{theorem}


\begin{figure}[t]
\centering
\begin{tikzpicture}[on grid,node distance=1.8 and 2.5]
\renewcommand{\thcite}[1]{Th.~\thref{#1}}
\node (AC-entails) {$\ABC^{\entails}$};
\node[below=2 of AC-entails] (SC-entails) {$\SBC^{\entails}$};
\node[below=of SC-entails] (SC-equiv) {$\SBC^{\equiv}$};
\node[left=of AC-entails] (AC-f) {$\op{(\ABC^\forw)}$};
\node[left=of SC-entails] (SC-f) {$\op{(\SBC^\forw)}$};
\node[left=of SC-equiv] (SC-fo) {$\op{(\SBC^{\forw\circ})}$};
\node[right=of AC-entails] (AC-b) {$\op{(\ABC^\back)}$};
\node[right=of SC-entails] (SC-b) {$\op{(\SBC^\back)}$};

\path
  (AC-f) edge[->] node[above] {\thcite{ab-categories}} (AC-entails)
  (AC-b) edge[->] node[above] {\thcite{ab-categories}} (AC-entails)
  (SC-f) edge[->] node[above] {\thcite{sb-categories}} (SC-entails)
  (SC-b) edge[->] node[above] {\thcite{sb-categories}} (SC-entails)
  (SC-fo) edge[->] node[above] {\thcite{sb-categories}} (SC-equiv)
  (AC-f) edge[->]  node[left,pos=.35] {$\op{\cN}$}
                   node[left,pos=.65] {(\thcite{ab-to-sb functors})}  (SC-f)
  (SC-fo) edge[->] node[left,pos=.65] {$\op{\cN}$}
                   node[left,pos=.35] {(\thcite{ab-to-sb functors})}  (SC-f)
  (AC-entails.230) edge[->]
                   node (N) {}
                   node[left,pos=.35] {$\cN$}
                   node[left,pos=.65] {(\thcite{ab-to-sb functors})} (SC-entails.130)
  (SC-entails.50) edge[->] 
                   node (A) {}
                   node[right,pos=.65] {$\cA$} 
                   node[right,pos=.35] {(\thcite{sb-to-ab functor})} (AC-entails.310)
  (SC-equiv) edge[->] (SC-entails)
  (AC-b) edge[->]  node[right,pos=.35] {$\op{\cN}$}
                   node[right,pos=.65] {(\thcite{ab-to-sb functors})} (SC-b);
				   
\path
  (A) edge[none] node {$\cong$} (N);
\end{tikzpicture}

\caption{Overview of categories and functors studied in this paper}
\flabel{categories}
\end{figure}

