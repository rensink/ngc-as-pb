\section{Conclusion and future work}
\slabel{conclusion}

\begin{itemize}
    \item Characterize the entailments that can be represented as morphisms, both for ab- and sb-conditions
    \item Several equivalent representation of the sam arrow-based condition as span-based one exist. Study norma forms.
\item Using cospans rather than spans to obtain ``light'' nested conditions. Possible, but first attempt suggest that spans are superior in terms of existence of a larger number of morphisms.
\item Other essentially equivalent definitions of Nested Application Condion exists. Explore whether our notion of morphism can be adapted to them, and if the transformation into span-based conditions is applicable as well.
\item By equipping conditions with suitable interfaces, study the algebraic structure of conditions, possibly identifying a suitable cartesian bicategory. This could provide the third structure (the combinatorial one) which together with the logical and the algebraic/categorical one presented in \cite{DBLP:journals/corr/abs-2404-18795} could lift to full FOL the triangular correspondence presented in \cite{DBLP:conf/csl/BonchiSS18} for the existential-conjunctive fragment.
\item Functors induced by shift relations
\item This is perfect as Related Work (I dropped it from the Introduction): In his work on Existential Graphs~\cite{roberts1973-the-existential-graphs-of-charles-s.-peirce}, Charles S. Peirce proposed a graphical representation of full FOL, equipped with some kinds of graph manipulations which represent sound deductions. The notion of graph is not formalized in a standard algebraic way, but certainly it would be interesting to try to equip them with a categorical structure as done for conditions in this paper.  
\item other related works (actually works on Nested ACs): Extensions of Nested Application Conditions  to arbitrary categories; Tableau (orejas, leen); satisfiability (konig et al.), Nested Conditions for Reactive Systems. 
\end{itemize}

