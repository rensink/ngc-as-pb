\subsection{Future work}
\slabel{future}

We see the results of this paper as providing only a start for the study into span-based conditions, giving rise to many natural follow-up questions. We name a few that have occurred to us.
%
\begin{itemize}
\item Is there any independent characterisation of the fragment of entailment that is explained by span-based condition morphisms? Do forward-shift and backward-shift morphisms explain disctinct fragments as they do in the arrow-based case? (Our provisional answer to the second question is: forward-shift morphisms are strictly more powerful than backward-shift ones. As to the first question: certain laws of FOL are certainly \emph{not} explained by morphisms, such as $\neg\neg\phi\equiv \phi$ and $\phi\vee \neg(\phi\wedge\phi)\equiv \phi\vee\psi$.)

\item There are many syntactically different (but semantically equivalent) representations for the same property. For instance, it can be shown that replacing any span by another with the same pushout gives rise to an equivalent condition. Is there a useful normal form for span-based conditions, preferably such that, if a morphism exists between two conditions, one also exists between their normal forms? (Our first investigation into this question has not een not encouraging; for instance, restricting spans to monic up-arrows does not reduce expressive power but does limit the existence of morphisms and hence reduce the explainable fragment of entailment. Alternatively, replacing the spans by cospans carrying the same information, thus getting rid of the irrelevant syntactical differences caused by pushout-equivalent spans, seems to break a number of necessary compositionality results.)

\item Do span-based conditions form an algebra for the constructions of FOL, given suitably defined operators? (Here we are quite hopeful: for instance, we believe that conjunction corresponds to the product and disjunction to the coproduct in our (forward-shift) category of span-based conditions, whereas negation is captured by ``pushing down'' a condition along an identity span. As for existential quantifitcation, we would like to show that this is left-adjoint to the shift operator, as in the case of cospan-based conditions \cite{Konig}\todo{was it left-or right-adjoint? Insert correct citation.})
\end{itemize}
%
Given the related-work discussion above, there are also some connections to be investigated further.
%
\begin{itemize}
\item By equipping conditions with suitable interfaces in the style of \cite{Bonchi}, enrich their algebraic structure, possibly identifying a suitable cartesian bicategory. This could provide the third structure (the combinatorial one) which together with the logical and the algebraic/categorical one presented in \cite{DBLP:journals/corr/abs-2404-18795} could lift to full FOL the triangular correspondence presented in \cite{DBLP:conf/csl/BonchiSS18} for the existential-conjunctive fragment.
\item Re-investigate the related work on cospan-based conditions in the light of our categories.
\item In his work on Existential Graphs~\cite{roberts1973-the-existential-graphs-of-charles-s.-peirce}, Charles S. Peirce proposed a graphical representation of full FOL, equipped with some kinds of graph manipulations which represent sound deductions. The notion of graph is not formalized in a standard algebraic way, but certainly it would be interesting to try to equip them with a categorical structure as done for conditions in this paper.  
\end{itemize}
