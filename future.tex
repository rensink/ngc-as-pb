\subsection{Future work}
\slabel{future}

\begin{itemize}
\item Characterize the (provable) entailments that can be represented as morphisms, both for ab- and sb-conditions. A theory of provable entailment: backward-shift evidence is subsumed by forward-shift evidence
\item Several equivalent representation of the same arrow-based condition as span-based one exist. Study normal forms. Spans with the same pushout are provably equivalent
\item Using cospans rather than spans to obtain ``light'' nested conditions. Possible, but first attempts suggest that spans are superior in terms of existence of a larger number of morphisms.
\item By equipping conditions with suitable interfaces, study the algebraic structure of conditions, possibly identifying a suitable cartesian bicategory. This could provide the third structure (the combinatorial one) which together with the logical and the algebraic/categorical one presented in \cite{DBLP:journals/corr/abs-2404-18795} could lift to full FOL the triangular correspondence presented in \cite{DBLP:conf/csl/BonchiSS18} for the existential-conjunctive fragment.
\item Re-investigate the related work on cospan-based conditions in the light of our categories
\item Canonical forms, e.g.\ the minimal (``thin'') span-based conditions, or mono up-arrows. We cannot restrict to these without losing provable entailment!
\item Constructions on span-based conditions, e.g.\ for conjunction (= product?) and disjunction (= coproduct?), negation (= pushing down one level), existential quantification (= back-shift?)
\item In his work on Existential Graphs~\cite{roberts1973-the-existential-graphs-of-charles-s.-peirce}, Charles S. Peirce proposed a graphical representation of full FOL, equipped with some kinds of graph manipulations which represent sound deductions. The notion of graph is not formalized in a standard algebraic way, but certainly it would be interesting to try to equip them with a categorical structure as done for conditions in this paper.  
\end{itemize}
