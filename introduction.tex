\section{Introduction}

Representing formulas of First-Order Logic (FOL) by graphs or more general graphical structures was explored in various areas of Theoretical Computer Science and Logics along the decades. 
A canonical example is represented by edge-labelled graphs, that can be regarded as an alternative syntax for formulas of a fragment of FOL including conjunction and existential quantification. 

Let $\Lab = \{\gl{a}, \gl{b}, \gl{c}, \ldots\}$ be a set of binary relation symbols, that we shall use also as edge labels. As an example, let $F$ be the graph $F = \twoedge{x}{a}{y}{b}{z}$. We consider it as a sound representation of the formula  $\Phi_F = \exists x,y,z\,.\, \gl{a}(x,y) \wedge \gl{b}(y,x)$, in the following sense: a graph $G$ satisfies $\Phi_F$ (or is a \emph{model} of $\Phi_F$) if and only if there is a graph morphism $h$ from $F$ to $G$. 
Now consider $F' = \oneedge{x}{a}{y}$, representing formula $\Phi_{F'} = \exists x,y\,.\, \gl{a}(x,y)$. Graph $F'$ has an obvious embedding morphism into $F$, $i:F' \to F$. Therefore every morphism $h: F \to G$ gives rise to a composed morphism $i;h: F'\to G$, implying that every graph that satisfies $\Phi_F$ also satisfies $\Phi_{F'}$, thus $\Phi_F$ \emph{entails} $\Phi_{F'}$ (written $\Phi_F \entails \Phi_{F'}$). It is worth noting that, in this elementary framework, graph morphisms can represent both a satisfaction relation (between formulas and models) and an entailment relation among formulas.

These concepts were exploited for example in \cite{DBLP:conf/stoc/ChandraM77} in the framework of relational database queries. Among other results, it is shown there that every \emph{conjunctive query} (a formula of the above fragment of FOL) has a natural model, a graph, and query inclusion is equivalent to the existence of a graph homomorphism between natural models. More recently, Bonchi et.al.~have proposed an algebraic presentation of \emph{graphical conjunctive queries}, having the same expressive power of the above conjunctive queries, but formalized as  string diagrams as arrows of cartesian bicategories. **** tring diagrams representing graphical conjunctive queries have been considered in \cite{DBLP:conf/csl/BonchiSS18}.

Extension of the above results to larger fragments of FOL are not as successful as the smaller fragment in capturing satisfaction and entailment. In his work on Existential Graphs~\cite{roberts1973-the-existential-graphs-of-charles-s.-peirce}, Charles S. Peirce proposed a graphical representation of full FOL, equipped with some graph manipulations which represent sound deductions. The approach was not formalized as we are used now within Category Theory, thus even if relationships could be explored we defer them.

Bonchi at al ... genralized \cite{DBLP:conf/csl/BonchiSS18}) to full FOL, but only for the string diagram representation, not for the graphical one...

GTS, NACS, Nested AC...

\begin{itemize}
    \item Edge labelled graphs  as existential - conjunctive formulas. Two properties: (1) existence of morphism to a graph G is satisfaction (2) looking at G as a formula, existence of morphism refletcs satisfaction.
    \item Example: DPO graph rewriting (LHS of a rule), conjunctive database queries \cite{DBLP:conf/stoc/ChandraM77}, representation of exists-and formulas using string diagrams (graphical conjunctive queries \cite{DBLP:conf/csl/BonchiSS18})
    \item In the realm of GTS's, rules can be equipped with NACS (refs's), of great practical interest. The match cannot be extended. 
    \item Extension to Nested Application Conditions (penneman, habel, arend) with a notion of satisfaction that makes them as expressive as FOL
    \item Extensions to arbitrary categories, tableau (orejas, leen), satisfiability (konig et al.) and as application conditions of general reactive systems
    \item  Goal of the paper: to propose a notion of morphism among Nested AC with a logical meaning (implying entailment), generalizing point (2) above. 
    
    By enriching Nested ACs and their morphisms with an algebraic structure...This could be the basis of a combinatorial presentaiton of FOL formulas, allowing to generalize to FOL the results by Pippo for regular fragment (ref's).
    
    After introducing morphims of Nested ACs implying satisfction preservation (logical consequence?), whihc form a category $\AC{R}$, we see that there are really few such morphisms, and identify the cause of such problem in the way Nested AC are traditionally defined. Thus we introduce a variation of the concept, span-based conditions, which are in a sense lighter. There is a full and faithful embedding of $\AC{R}$ into $SBAC(R)$, but each FOL formula has many more representative as span-based one, so we can discuss in a richer setting the existence of morphisms, and othher constructions....

\end{itemize}
    