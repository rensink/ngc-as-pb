\section{Morphisms of arrow-based conditions}
\slabel{ab-morphisms}

The notion of \emph{morphism} of nested conditions has not received much attention in the literature. When considered at all, such as in \cite{bchk:conditional-reactive-systems,sksclo:coinductive-techniques-for-satisfiability}, they are essentially based on the semantics in terms of satisfaction. Indeed, entailment establishes a preorder over conditions. Let us denote the resulting category $\ABC^{\entails}$.

The question that we address in this section is to establish a meaningful \emph{structural} notion of condition morphism. That is, given the fact that an ab-condition is essentially a diagram in the category $\bC$, a structural morphism from $b$ to $c$ consists of arrows between objects of $b$ and objects of $c$ satisfying certain commutativity conditions. For morphisms to be meaningful, they should certainly only exist where there is entailment; in other words, we require there to be a functor to $\ABC^{\entails}$. As we will see, this requirement implies that it is not enough to consider only arrows from objects of $b$ \emph{to} objects of $c$, since in our definition of satisfaction, the direction of entailment flips at every subsequent nesting level. We also observe that the existence of a functor to $\ABC^{\entails}$ means that we only consider morphisms between conditions with the same root.

The underlying principle for a morphism $m$ from $b=(R,p^b_1\ccdots p^c_{|b|})$ to $c=(R,p^c_1\ccdots p^c_{|c|})$ is that it must identify, for every $b$-branch $p^b_i$, a $c$-branch $p^c_j$ that entails it. Entailment of branches, in turn, is captured by the existence of an arrow $v_i$ from $P^c_j$ to $P^b_i$ (hence going in the \emph{opposite} direction of $m$) such that there is (recursively, but in the opposite direction) a morphism between the subconditions $c_j$ and $b_i$; however, this sub-morphism can only exist ``modulo" the morphism $v_i$ between their roots. The task of the next subsection is to make the concept of ``modulo $v_i$" precise.

\subsection{Root shifting}
\slabel{shift-morphisms}

In order to relate conditions at different roots, we introduce a special kind of mapping called a \emph{root shifter}, which will later\todo{AR: Add link to where this is done} turn out to be a functor between categories of conditions over different roots. For now, we note that the purpose of root shifters is to relate the models of a condition $b\in \AC(A)$ to those of a condition $c\in \AC(B)$, given an arrow $v:A\to B$ between their roots (the existence of which implies that the model sets are typically disjoint). To be more precise, we will require that \emph{every $v$-prefixed model of $b$ induces a model for $c$}: that is, if $v;g$ is a model of $b$, then $g$ is a model of $c$.

This can be achieved in essentially two ways, both on the basis of $v$: by constructing $c$ from $b$ (which we will call \emph{forward shifting}) or by constructing $b$ from $c$ (which we will call \emph{backward shifting}).

We will first define how to shift arrows (where it is called source shifting) and then extend this in a natural way to conditions (where it is called root shifting).

\begin{definition}[source shifter]\dlabel{ab-source shifter}
\begin{itemize}[topsep=\smallskipamount]
\item Let $X,Y$ be objects. A \emph{source shifter} $\cS$ from $X$ to $Y$ is a function from $X$-sourced to $Y$-sourced arrows, such that for all arrows $a\of X\to Y, t\of Y\to Z$:
%
\begin{equation}\eqlabel{shifters are functors}
\cS(a;t) = \cS(a);t \enspace.
\end{equation}
%
\item Now let $v\of A\to B$ an arrow. A \emph{forward source shifter $\cS$ for $v$} is a source shifter from $A$ to $B$ such that for all $a\of A\to C$, $g\of B\to G$ and $h\of C\to G$:
%
\begin{equation}\eqlabel{forward condition}
v;g = a;h \text{ implies } g = \cS(a);h \enspace.
\end{equation}
%
\item A \emph{backward source shifter $\cS$ for $v$} is a source shifter from $B$ to $A$ such that for all $a\of B\to C$, $g\of B\to G$ and $h\of C\to G$:
%
\begin{equation}\eqlabel{backward condition}
v;g = \cS(a);h \text{ implies } g = a;h \enspace.
\end{equation}
\end{itemize}
\end{definition}
%
If $\cS$ is known to be a forward shifter, we sometimes use $\cF$ to denote it; similarly, we use $\cB$ to denote backward shifters. Condition \eqref{shifters are functors} is actually quite strong, since it holds for any arrow $a$ and $t$, in particular also for the case that $a=\id_X$ (and hence $Y=X$). This leads to the following observation.
%
\begin{lemma}\llabel{ab-source shifters}
Let $\cS$ be a mapping from $X$-sourced to $Y$-sourced arrows, and let $v\of Y\to X$ be an arrow. The following statements are equivalent:
\begin{enumerate}[topsep=\smallskipamount,label=(\alph*),ref=(\alph{enumi})]
\item\label{shifter-1} $\cS(a)=v;a$ for all $a\of X\to Z$;
\item\label{shifter-2} $\cS$ is a source shifter from $X$ to $Y$ such that $\cS(\id_X)=v$.
\end{enumerate}
\end{lemma}
%
\begin{fullorname}
\begin{proof}
\begin{itemize}[topsep=\smallskipamount]
\item \emph{\ref{shifter-1} implies \ref{shifter-2}.}
%
Assume that \ref{shifter-1} holds. To show that  $\cS$ is a source shifter, note that  \eqcite{shifters are functors} holds by $\cS(a;t)=v;a;t=\cS(a);t$. Furthermore, $\cS(\id_X)=v;\id_X=v$.

\item \emph{\ref{shifter-2} implies \ref{shifter-1}.}
%
Assume that \ref{shifter-2} holds. Let $a:X\to Z$; by \eqref{shifters are functors}, it follows that $\cS(a)=\cS(\id_X;a)=\cS(\id_X);a=v;a$.\qed
\end{itemize}
\end{proof}
\end{fullorname}
%
Knowing that source shifters from $X$ are entirely determined by $\cS(\id_X)$ gives us a basis for their construction. Before that, let us introduce some categorical terminology. An arrow $r: A \to B$ is a \emph{split epi(morphism)} if there is an arrow $m:B\to A$ such that $m;r = id_B$. In this case, arrow $m$ which satisfies the dual condition is a 
\emph{split mono(morphism)}. We also say that $r$ is a \emph{retraction} of $m$, and that $m$ is a \emph{section} of $r$.

The following proposition characterises the (forward and backward) root shifters that exist in general.
%
\begin{proposition}[known source shifters]\plabel{ab-source shifters}
Let $v\of A\to B$ be an arrow.
\begin{enumerate}[topsep=\smallskipamount]
\item The following three statements are equivalent.
\begin{enumerate}[label=(\alph*),ref=\theenumi(\theenumii)]
\item\label{forward-1} $\cS$ is a forward source shifter for $v$;
\item\label{forward-2} $\cS$ is a source shifter from $A$ to $B$ such that $\cS(v)=\id_B$;
\item\label{forward-3} $\cS$ is a mapping from $A$-sourced to $B$-sourced arrows such that $s=\cS(\id_A)$ is a section of $v$ and $\cS(a)=s;a$ for all $a\of A\to C$.
\end{enumerate}
\item If $\cS$ is a source shifter such that $\cS(\id_B)=v$, then $\cS$ is a backward source shifter for $v$ if and only if $v$ is epi.
\end{enumerate}
\end{proposition}
%
\begin{fullorname}
\begin{proof}
\begin{enumerate}[topsep=\smallskipamount]
\item
We show $\ref{forward-1} \Rightarrow \ref{forward-2} \Rightarrow \ref{forward-3} \Rightarrow \ref{forward-1}$.
\begin{itemize}
\item \emph{\ref{forward-1} implies \ref{forward-2}.}
%
Assume that \ref{forward-1} holds. By definition, $\cS$ is a source shifter from $A$ to $B$. Now consider \eqref{forward condition} with $a=v$ and $g=h=\id_B$: then $v;g=a;h$, hence $\id_B=g = \cS(a);h=\cS(a)$.

\item \emph{\ref{forward-2} implies \ref{forward-3}.}
%
Assume that \ref{forward-2} holds. By definition, $\cS$ is a mapping from $A$-sourced arrows to $B$-sourced arrows; moreover, if $s=\cS(\id_A)$ then $s;v=\cS(\id_A);v=\cS(\id_A;v)=\cS(v)=\id_B$, proving that $s$ is a section of $v$. Moreover, $\cS(a)=\cS(\id_A;a)=\cS(\id_A);a=s;a$.

\item \emph{\ref{forward-3} implies \ref{forward-1}.}
%
Assume that \ref{forward-3} holds. To show \eqref{shifters are functors}, note that $\cS(a;t)=s;a;t=\cS(a);t$. To show \eqref{forward condition}, consider $a,g,h$ such that $v;g=a;h$; then $g=\id_A;g=s;v;g=s;a;h = \cS(a);h$.
\end{itemize}

\item Assume that $\cS$ is a source shifter such that $\cS(\id_B)=v$.
\begin{itemize}
\item \emph{$v$ epi implies that $\cS$ is a backward source shifter for $v$.} To show \eqref{backward condition}, let $a,g,h$ be such that $v;g=\cS(a);h=v;a;h$; then $g=a;h$ because $v$ is epi.

\item \emph{$\cS$ a backward source shifter for $v$ implies that $v$ is epi.} Let $g,h:B\to C$ be such that $v;g=v;h$, and let $a=\id_B$; then $v;g=\cS(a);h$, hence \eqref{backward condition} implies $g=a;h=\id_B;h=h$. It follows that $v$ is epi.\qed
\end{itemize}
\end{enumerate}
\end{proof}
\end{fullorname}
%
It follows that $v$ has precisely one forward source shifter for every section $s$ of $v$ (meaning, among other things, that $v$ must be a split epi --- hence a fortiori an epi --- to have any forward shifter at all). As for backward source shifters, if $\cS(\id_B)=v$ (which is in general the only known arrow from $A$ to $B$), then $v$ has a backward source shifter if and only if $v$ is epi. We denote these source shifters by $\cF_v^s$ and $\cB_v$, respectively. Their working is visualised in \fcite{ab-source shifters}. \fcite{ex-ab-shifters} shows some examples.

\begin{figure}[t]
\centering
\begin{tikzpicture}[on grid]
  \node (A2) {$A$};
  \node (B2) [right=2 of A2] {$B$};
  \node (B2') [below left=1 and .5 of B2] {$B$};
  \node (C2) [below right=2 and 1 of A2] {$C$};

  \path (A2) edge[->] node[left] {$a$} (C2)
        (B2') edge[->] node[right] {$\id_B$} (B2) 
        (B2') edge[->] node[right] {$\cF_v^s(a)$} (C2)
		(A2) edge[->] node [above] {$v$} (B2)
        (B2') edge[->] node[below] {$s$} (A2);
\end{tikzpicture}
\qquad
\begin{tikzpicture}[on grid]
  \node (A1) {$A$};
  \node (B1) [right=2 of A1] {$B$};
  \node (C1) [below right=2 and 1 of A1] {$C$};
  
  \path (A1) edge[->] node[left] {$\cB_v(a)$} (C1)
        (B1) edge[->] node[right] {$a$} (C1)
		(A1) edge[->>] node[above] {$v$} (B1);
\end{tikzpicture}

\caption{Forward and backward source shifting for $v:A\to B$}
\flabel{ab-source shifters}
\end{figure}

\begin{figure}
\centering
\newcommand{\Cgraph}{\mygraph{
    \node[n] (1) {$y$};
    \node[n] (2) [right=of 1] {$x$};
    \node[n] (3) [right=of 2] {$z$};
    \path (1) edge[<-] node[above] {\la} (2)
          (2) edge[->,bend left] node[near start,above] {\la} (3)
          (3) edge[->,bend left] node[near start,below] {\lb} (2);
    }
}%
%
\begin{tikzpicture}[on grid]
  \node[graph] (A) {\spangraph{y}{a}{x}{a}{z}};
  \node [above right=0 and 0 of A.north west] {$A$};
  \node[graph,below=2 of A] (C) {\Cgraph};
  \node [above right=0 and 0 of C.north west] {$C$};
  \node[graph] (B) [right=2.3 of A] {\oneedge{x'}{a}{z'}};
  \node [above left=0 and 0 of B.north east] {$B$};

  \path (A) edge[->] node[left] {$a$} (C);

  \path (A) edge[->,bend left] node[above]
            {$v \left\{\mapping{
			x & x' \\
			y & z' \\
			z & z'}\right.$}
	    (B);

  \path (B) edge[->] node[above] {$s_1$}
                     node[below]
            {$\mapping{
			x' & x \\
			z' & y}$}
	    (A);

  \path (B.south) edge[->] node[near start,below right,inner sep=0pt] {$\cF_v^{s_1}(a) $}  
                           node[near end,below right=-.1] 
            {$\mapping{
			x' & x \\
			z' & y}$}
		(C.north east);
\end{tikzpicture}%
\enspace
\begin{tikzpicture}[on grid]
  \node[graph] (A) {\spangraph{y}{a}{x}{a}{z}};
  \node [above right=0 and 0 of A.north west] {$A$};
  \node[graph,below=2 of A] (C) {\Cgraph};
  \node [above right=0 and 0 of C.north west] {$C$};
  \node[graph] (B) [right=2.3 of A] {\oneedge{x'}{a}{z'}};
  \node [above left=0 and 0 of B.north east] {$B$};

  \path (A) edge[->] node[left] {$a$} (C);

  \path (A) edge[->,bend left] node[above]
            {$v \left\{\mapping{
			x & x' \\
			y & z' \\
			z & z'}\right.$}
	    (B);

  \path (B) edge[->] node[above] {$s_2$} 
                     node[below]
            {$\mapping{
			x' & x \\
			z' & z}$}
	    (A);

  \path (B.south) edge[->] node[near start,below right,inner sep=0pt] {$\cF_v^{s_2}(a)$}
                           node[near end,below right=-.1] 
            {$\mapping{
			x' & x \\
			z' & z}$}
		(C.north east);
\end{tikzpicture}%
\enspace
\begin{tikzpicture}[on grid]
  \node[graph] (A) {\spangraph{y'}{a}{x'}{a}{z'}};
  \node [above right=0 and 0 of A.north west] {$A$};
  \node[graph] (B) [right=1.9 of A] {\oneedge{x}{a}{z}};
  \node [above left=0 and 0 of B.north east] {$B$};
  \node[graph,below left=2 and .4 of B] (C) {\Cgraph};
  \node [above left=0 and 0 of C.north east] {$C$};

  \path (B) edge[->] node[right] {$a$} (C.north -| B);

  \path (A) edge[->,bend left] node[above]
            {$v \left\{\mapping{
			x' & x \\
			y' & z \\
			z' & z}\right.$}
	    (B);

  \path (A.south) edge[->] node[near start,below left=-.1] {$\cB_v(a)$}
                           node[below left] 
            {$\mapping{
			x' & x \\
			y' & z \\
			z' & z}$}
		(C);
\end{tikzpicture}


\caption{Forward source shifters $\cF_v^{s_1}$ and $\cF_v^{s_2}$ and backward source shifter $\cB_v$ for an arrow $v:A\to B$. (All arrows $a$ are the identity on nodes.)}
\flabel{ex-ab-shifters}
\end{figure}

Source shifting is extended in a straightforward manner to conditions, where it is called \emph{root shifting}. To root-shift a condition, we only have to source-shift the arrows of its branches: the subconditions remain the same. This also means that for zero-width conditions, there is essentially nothing to be done. Hence we can either extend a source shifter $\cS$ from $X$ to $Y$ to a mapping $\bar\cS\of \AC X\to \AC Y$ over conditions of arbitrary width, or use the special \emph{trivial} root shifter $\cI_{X,Y}$ for zero-width conditions, defined as follows:
\[\begin{array}{rrl}
\bar\cS: & c & \mapsto (Y,(\cS(a^c_1),c_1)\ccdots (\cS(a^c_{|w|}),c_{|w|})) \\
\cI_{X,Y}: & (X,\epsilon) & \mapsto (Y,\epsilon) \enspace.
\end{array}\]

\begin{definition}[root shifter]\dlabel{ab-root shifter}
Let $X,Y$ be objects and $v\of A\to B$ an arrow.
\begin{itemize}[topsep=\smallskipamount]
\item A \emph{root shifter} $\cS$ from $X$ to $Y$ is a partial function from $\AC X$ to $\AC Y$ such that either $\cC=\bar\cS$ for some source shifter from $X$ to $Y$, or $\cC=\cI_{X,Y}$.

\item A \emph{forward root shifter} $\cC$ for $v$ is a root shifter from $A$ to $B$ such that either $\cC=\bar\cF$ for some forward source shifter $\cF$, or $\cC=\cI_{A,B}$.

\item A \emph{backward root shifter} $\cC$ for $v$ is a root shifter from $B$ to $A$ such that either $\cC=\bar\cB$ for some backward source shifter $\cB$, or $\cC=\cI_{B,A}$.
\end{itemize}
\end{definition}
%
It should be noted that \emph{all} root shifters $\bar\cS$ with $\cS$ a source shifter from $X$ to $Y$ behave as $\cI_{X,Y}$ on zero-width conditions; however, $\cI_{X,Y}$ is a root shifter for \emph{any} $v$, whereas, as we have seen in \dcite{ab-source shifter}, forward and backward source shifters only exist for certain $v$. This makes it useful to distinguish this special case.

\begin{fullorname}[\lname{ab-root shifter}]
The following fact, which follows from \lcite{ab-source shifters} for source shifters, will be very useful in reasoning about root shifters.
%
\begin{lemma}[root shifter property]\llabel{ab-root shifters}
Let $X,Y$ be objects. For any non-trivial root shifter $\cC$ from $X$ to $Y$, there is an arrow $v:Y\to X$ such that, for all $c\in \AC X$ on which $\cC$ is defined, $\cC(c)= (Y,(v;a^c_1,c_1)\ccdots (v;a^c_{|c|},c_{|c|}))$.
\end{lemma}
%
\begin{proof}
If $\cC$ is non-trivial, then $\cC=\bar\cS$ for some source shifter. Let $v=\cS(\id_X)$; then \lcite{ab-source shifters} implies $\cS(a)=v;a$ for any $A$-sourced $a$.\qed
\end{proof}
\end{fullorname}
%
By ignoring from now on the notational distinction between $\cS$ and $\bar\cS$, we reuse $\cF$ and $\cB$ to range over forward and backward root shifters. The action of a root shifter is visualised in \fcite{root shifting}.
%
\begin{figure}[t]
\centering
\begin{tikzpicture}[on grid]
  \node (A) {$A$};
  \node (B) [right=3 of A] {$B$};
  \tri[green]{A}{3.5}{4}{3.7}{};
  \tri[blue]{B}{3.5}{-.3}{3.7}{};
  \node (P1) [below right=2.2 and .8 of A,inner sep=1] {};%{$P_1$};
  \node (Pn) [below right=2.2 and 2.2 of A,inner sep=1] {};%{$P_n$};
  \node [below right=2.2 and 1.5 of A] {$\cdots$};

  \tri{P1}{1.2}{.6}{1}{$c_1$}
  \tri{Pn}{1.2}{.4}{1}{$c_n$}
  
  \path [allcolor={\darker{green}}]
        (A.295) edge[->] node[left,pos=.45] (ab1) {$a_1$} (P1)
                edge[->] node[above right,pos=.2] (abw) {$a_w$} (Pn)
		;
  
  \path [draw=blue,text=blue,>/.tip={Latex[blue]}]
        (B.245) edge[->] node[above left,pos=.2,inner sep=1] (ac1) {$\cS_1(a_1)$} (P1)
                edge[->] node[right,pos=.45] (acw) {$\cS_w(a_w)$} (Pn)
		;
\end{tikzpicture}

\caption{Root shifting using a source shifter $\cS$: green is shifted to blue}
\flabel{root shifting}
\end{figure}
%
The following is the crucial preservation property of root shifters:
%
\begin{proposition}[root shifters preserve models]\plabel{ab-root shifters preserve}
Let $v:A\to B$.
\begin{enumerate}[topsep=\smallskipamount]
\item If $c\in \AC A$ and $\cF$ is a forward root shifter for $v$ that is defined on $c$, then $v;g\sat c$ implies $g\sat \cF(c)$.
\item If $c\in \AC B$ and $\cB$ is a backward root shifter for $v$ that is defined on $c$, then $v;g\sat \cB(c)$ implies $g\sat c$.
\end{enumerate}
\end{proposition}
%
\begin{fullorname}
\begin{proof}
For zero-width conditions, both clauses are vacuously true. In the remainder assume $|c|>0$ (and hence the root shifters are non-trivial). Let $v:A\to B$.
\begin{enumerate}
\item Let $b=\cF(c)$; then $a^b_i=\cF(a^c_i)$ for all $1\leq i\leq |c|$. Assume $v;g\sat c$ due to responsible branch $p^c_i$ and witness $h$; i.e., $v;g=a^c_i;h$. \eqcite{forward condition} then implies $g=\cF(a);h$, hence $p^b_i$ is a responsible branch and $h$ a witness for $g\sat b$.
\item Let $b=\cB(c)$; then $a^b_i=\cB(a^c_i)$ for all $1\leq i\leq |c|$. Assume $v;g\sat b$ due to responsible branch $p^b_i$ and witness $h$; i.e., $v;g=\cB(a^c_i);h$. \eqcite{backward condition} then implies $g=a;h$, hence $p^c_i$ is a responsible branch and $h$ a witness for $g\sat c$.
\qed
\end{enumerate}
\end{proof}
\end{fullorname}
%
Source and root shifters compose, in the following way.
%
\begin{proposition}[shifters compose]\plabel{ab-shifters compose}
\begin{enumerate}[topsep=\smallskipamount]
\item If $\cU$ is a source [root] shifter from $X$ to $Y$ and $\cV$ a source [root] shifter from $Y$ to $Z$, then $\cU;\cV$ (understood as partial function composition) is a source [root] shifter from $X$ to $Z$. 
\item If $\cU$ and $\cV$ are forward source [root] shifters for $u: X\to Y$ and $v: Y \to Z$, respectively, then $\cU;\cV$ is a forward source [root] shifter for $u;v$.
\item Dually, if $\cU$ and $\cV$ are backward source [root] shifters for $u: Y\to X$ and $v: Z \to Y$, respectively, then $\cU;\cV$ is a backward source [root] shifter for $v;u$.
\end{enumerate}
\end{proposition}

\begin{fullorname}
\begin{proof}
We first prove the case for source shifters.
\begin{enumerate}
\item To prove \eqcite{shifters are functors} for $\cU;\cV$ consider:
\[ (\cU;\cV)(a;t) = \cV(\cU(a;t))= \cV(\cU(a);t)=\cV(\cU(a));t=(\cU;\cV)(a);t \enspace. \]

\item In case $\cU$ and $\cV$ are forward, to prove \eqcite{forward condition} assume $u;v;g=a;h$. By applying \eqcite{forward condition} for $\cU$ and $\cV$ it follows that $v;g=\cU(a);h$ and hence $g=\cV(\cU(a));h=(\cU;\cV)(a)$.

\item In case $\cU$ and $\cV$ are backward, to prove \eqcite{backward condition} for $\cU;\cV$ assume $v;u;g=(\cU;\cV)(a);h=\cV(\cU(a));h$. By applying \eqcite{backward condition} for $\cV$ and $\cU$ it follows that $u;g=\cU(a);h$ and hence $g=a;h$.
\end{enumerate}
%
Now assume $\cU$ and $\cV$ are root shifters. If both are non-trivial, then so is $\cU;\cV$ and the result follows by a straightforward pointwise argument. If, on the other hand, either $\cU$ or $\cV$ is trivial, then in fact so is $\cU;\cV$, which immediately establishes the result.
\qed
\end{proof}
\end{fullorname}

\subsection{Structural morphisms}

The following defines structural morphisms, based on root shifters, over ab-conditions with the same root.

\begin{definition}[arrow-based condition morphism]\dlabel{ab-morphism}
  Let $b,c \in \AC{R}$. A \emph{forward-shift [backward-shift] condition morphism} $m: b \to c$ is a pair $(o,(v_1,m_1)\ccdots (v_{|b|},m_{|b|}))$ where
  \begin{itemize}[topsep=\smallskipamount]
  \item $o:[1,|b|]\to[1,|c|]$ is a function from $b$'s branches to $c$'s branches;
  \item for all $1\leq i\leq |b|$, $v_i:P^c_{o(i)}\to P^b_i$ is an arrow from the pattern of $p^c_{o(i)}$ to that of $p^b_i$ such that $a^c_{o(i)};v_i=a^b_i$;
  \item\emph{Forward shift}: for all $1\leq i\leq |b|$, there is a forward root shifter $\cF_i$ for $v_i$ such that $m_i:\cF_i(c_{o(i)})\to b_i$ is a forward-shift morphism.
  \item\emph{Backward shift}: for all $1\leq i\leq |b|$, there is a backward root shifter $\cB_i$ for $v_i$ such that $m_i:c_{o(i)}\to \cB_i(b_i)$ is a backward-shift morphism.
  \end{itemize}
\end{definition}
%
We use the same notational conventions as for conditions: $m$ has width $|m|$ and depth $\depth(m)$, and the components of a morphism $m$ are denoted $o^m$, $v^m_i$ and $m_i$ for all $1\leq i\leq |m|$. \fcite{shifted-morphism} visualises the principle of forward- and backward-shift morphisms.
%
\begin{figure}[t]
\centering
\begin{tikzpicture}[on grid]
\node (R) {$R$};
\tri[green]{R}{3.2}{-.3}{2.8}{};
\node [below left=2.8 and 2.4 of R,text=\darker{green}] {$b$};
\node (Pbi) [below left=1.6 and 1.0 of R] {};
\path [allcolor={\darker{green}}] (R) edge[->] node[left] {$a^b_i$} (Pbi);
\tri[green]{Pbi}{1.5}{.2}{1.4}{};
\node (bi) [below left=.8 and .25 of Pbi,text=\darker{green}] {$b_i$};

\tri[blue]{R}{3.2}{3.3}{2.8}{};
\node [below right=2.8 and 2.4 of R,text=blue] {$c$};
\node (Pci) [below right=1.6 and 1.0 of R] {};
\path [allcolor=blue] (R) edge[->] node[right] {$a^c_{o(i)}$} (Pci);
\tri[blue]{Pci}{1.5}{1.2}{1.4}{};
\node (ci) [below right=.8 and .25 of Pci,text=blue] {$c_{o(i)}$};

\path (Pci) edge[->] node[above] {$v_i$} (Pbi);

\tri[red]{Pbi}{1.5}{3.2}{1.4}{};
\node [left=1.1 of ci,red,inner sep=1pt] (Ci) {$\cC_i(c_{o(i)})$};

\path [->,bend left=80,looseness=2,below,dotted]
      (ci) edge[looseness=2.5] node {$\cC_i$} (Ci)
      (Ci) edge node {$m_i$} (bi)
	  ;
\end{tikzpicture}
%
\qquad
%
\begin{tikzpicture}[on grid]
\node (b) {$b$};
\node (Rb) [below=.7 of b] {$R$};
\node (Pb1) [below left=of Rb] {};
\node [below left=.8 and .4 of Rb] {$\cdots$};
\node (Pbi) [below=1.8 of Rb] {$P^b_i$};
\node [below right=.8 and .4 of Rb] {$\cdots$};
\node (Pbn) [below right=of Rb] {};
\node (bi) [triangle,below=.20 of Pbi.center] {$b_i$};

\path (Rb) edge[->] (Pb1)
      (Rb) edge[->] (Pbn)
	  (Rb) edge[->] (Pbi);

\node (c) [right=3 of b] {$c$};
\node (Rc) [below=.7 of c] {$R$};
\node (Pc1) [below left=of Rc] {};
\node [below left=.8 and .4 of Rc] {$\cdots$};
\node (Pci) [below=1.8 of Rc] {$P^c_{o(i)}$};
\node [below right=.8 and .4 of Rc] {$\cdots$};
\node (Pcn) [below right=of Rc] {};

\node (bi-top) [below left=.20 and .01 of Pci.center,inner sep=0] {};
\node (bi-right) [left=.1 of bi-top,inner sep=0] {};
\node (bi-outer) [left=1.3 of bi-right,inner sep=0] {};
\node (bi-mid) [left=.7 of bi-right] {};

\draw (bi-top.center) -- (bi-right.center |- bi.south)
                      -- (bi-outer.center |- bi.south)
					  -- (bi-top.center);

\draw[draw=none] (bi-right |- bi) -- node[pos=.3] {$b'_i$} (bi-outer |- bi);

\node (ci-top) [below right=.20 and .01 of Pci.center,inner sep=0] {};
\node (ci-inner) [right=.1 of ci-top,inner sep=0] {};
\node (ci-outer) [right=1.3 of ci-inner,inner sep=0] {};
\node (ci-mid) [right=.7 of ci-inner] {};

\draw (ci-top.center) -- (ci-inner.center |- bi.south)
                      -- (ci-outer.center |- bi.south)
					  -- (ci-top.center);

\draw[draw=none] (ci-inner |- bi) -- node (co') [pos=.3] {$c_{o(i)}$} (ci-outer |- bi);

\path (Rc) edge[->] (Pc1)
      (Rc) edge[->] (Pcn)
	  (Rc) edge[->] (Pci);

\path (Rb) edge[<->] node[above] {$\id$} (Rc)
      (Pci) edge[->] node[above] {$v_i$} (Pbi)
      (ci-mid |- bi.south) edge[->,bend left=40] node[below] {$m_i$} (bi-mid |- bi.south)
	  (bi) edge[dotted,->] node[above] {$\back{\cC_i}$} ++ (1.9,0);	  
\end{tikzpicture}

\caption{Visualiation of forward-shift and backward-shift morphisms}
\flabel{shifted-morphism}
\end{figure}
%
The next result states that the existence of a morphism between two conditions $m: b \to c$ provides evidence that $b\entails c$.

\begin{proposition}[ab-condition morphisms preserve models]
  \plabel{ab-morphisms preserve models}
Let $b,c \in \AC{R}$  be arrow-based conditions. If $m:b\to c$ is a forward-shift or a backward-shift morphism, then $g\sat b$ implies $g \sat c$ for all arrows $g:R\to G$. 
\end{proposition}
%
\begin{fullorname}
\begin{proof}
Let $m = (o,(v_1,m_1)\ccdots (v_{|b|},m_{|b|}))$ and assume by induction hypothesis that for all $1\leq j\leq |b|$ the property holds for $m_j$. Let $p^b_i$ be the responsible branch and $h: P^b_i \to G$ the witness for $g\sat b$. 
Hence we have $(\dagger)\, g=a^b_i;h$ and $(\ddagger)\, h \nsat b_i$.  
We show that $g \sat c$, with responsible branch $p^c_{o(i)}$ and witness $h$. In fact, we immediately have  $a^c_{o(i)}; v_i ;h = a^b_i ; h \stackrel{(\dagger)}{=} g$. It remains to show that $v_i ;h \nsat c_{o(i)}$, for which we consider separately the two cases.
\begin{enumerate}
\item Let $m$ be a forward-shift morphism. Then for all $1\leq j\leq |b|$ we have that   $m_j:\cF_j(c_{o(j)}) \to b_j$ is a forward-shift morphism, where $\cF_j$ is a forward root shifter for $v_j$. If (ad absurdum) $v_i ;h \sat c_{o(i)}$, by clause 1 of \pcite{ab-root shifters preserve} we would have $h \sat \cF_i(c_{o(i)})$, and by induction hypothesis on $m_i$ also $h \sat b_i$, contradicting $(\ddagger)$.

\item Let $m$ be a backward-shift morphism. Then for all $1\leq j\leq |b|$ we have that  $m_j:c_{o(j)}\to \cB_j(b_j)$ is a backward-shift morphism, where $\cB_j$ is a backward root shifter of width $|b_j|$.
Suppose (ad absurdum) that $v_i ;h \sat c_{o(i)}$. Then by the induction hypothesis on $m_i$ we also have $v_i ;h \sat \cB_i(b_i)$, and by clause 2 of \pcite{ab-root shifters preserve} we  have $h \sat b_i$, contradicting $(\ddagger)$.
\qed
\end{enumerate}
\end{proof}
\end{fullorname}

\begin{example}\exlabel{ab-morphisms}
Going back to \fcite{ab-conditions}, though (as noted before) $c_1\entails c_2$, there does not exist a morphism $m\of c_1\to c_2$. A mapping $v_1$ does exist from $c_{21}$ to $c_{11}$; however, no mapping $v_{11}$ can be found to $c_{211}$ from either $c_{111}$ or $c_{112}$, nor a mapping $v_{12}$ to $c_{212}$.

On the other hand, $c_2\entails c_3$ has evidence in the form of a morphism $m\of c_2\to c_3$, with the identity mapping $v_1\of c_{31}$ to $c_{21}$ (in combination with either the forward shifter $\cF_{v_1}^{v_1}$ or the backward shifter $\cB_{v_1}$) and a corresponding sub-morphism $m_1$ with mapping $v_{12}:c_{212}\to c_{321}$ (in combination with the trivial root shifter).
\qed
\end{example}
%
\pcite{ab-morphisms preserve models} can be phrased more abstractly as the existence of a functor from a category of arrow-based conditions and their (forward-shift or backward-shift) morphims to $\ABC^{\entails}$. An important step to arrive at that result is to show that morphims compose. This, in turn, depends on the preservation of morphisms by root shifters.

\begin{lemma}[root shifters preserves morphisms]
             \llabel{ab-root shifters preserve morphisms}
Let $b,c \in  \AC{X}$ and let $\cC$ be a root shifter from $X$ to $Y$ that is defined on $b$ and $c$. If $m:b\to c$, then also $m:\cC(b)\to\cC(c)$. 
\end{lemma}
%
\begin{fullorname}
\begin{proof}
If $\cC$ is trivial, the result is immediate. Otherwise, assume $m = (o,(v_1,m_1)\cdots(v_{|b|},m_{|b|}))$ and let $t$ be as in \lcite{ab-root shifters}. The only new proof obligation for $m$ to be an ab-morphism from $b$ to $c$ is that, for all $1\leq i\leq |b|$, $t;a^c_{o(i)};v_i=t;a^b_i$. This follows immediately from $a^c_{o(i)};v_i=a^b_i$.\qed
\end{proof}
\end{fullorname}
%
Composition of ab-condition morphisms can be defined inductively, and is independent of their forward or backward nature. We also (inductively) define arrow-based identity morphisms. Let $b,c,e$ be ab-conditions and $m:b\to c,n:c\to e$ ab-morphisms, with $|b|=w$.
%
\begin{eqnarray}
\id_b & =
  & (\id_{[1,w]},(\id_{P^b_1},\id_{b_1})\cdots 
                 (\id_{P^b_w},\id_{b_w}))
  \eqlabel{id-morphism} \\
m;n & =
  & (o^m;o^n,(v^n_{o^m(1)};v^m_1,n_{o^m(1)};m_1)\cdots 
              (v^n_{o^m(w)};v^m_w,n_{o^m(w)};m_w)) \enspace.
 \eqlabel{morphism composition}
\end{eqnarray}
%
Identities are morphisms, and morphism composition is well-defined.

\begin{lemma}[identities and composition of ab-morphisms]\llabel{ab-morphisms compose}
Let $b,c,e$ be ab-conditions and $m\of b\to c, n\of c\to e$ ab-condition morphisms.
\begin{enumerate}[topsep=\smallskipamount]
\item $\id_b$ is both a forward- and a backward-shift morphism from $b$ to $b$;
\item If $m$ and $n$ are forward-shift morphisms, then $m;n$ is a forward-shift morphism from $c$ to $e$;
\item If $m$ and $n$ are backward-shift morphisms, then $m;n$ is a backward-shift morphism from $c$ to $e$.
\end{enumerate}
\end{lemma}
%
\begin{fullorname}
\begin{proof}
Let $w=|b|$.
%
\begin{enumerate}[topsep=\smallskipamount]
\item By induction on the depth of $b$. For all $1\leq i\leq w$, $a^b_i;\id_{P^b_i}=a^b_i$. Note that $\rF{\id}{\id}$ and $\cB_\id$ (with $\id=\id_{P^b_i}$) are, respectively, forward and backward root shifters, both acting as the identity on $P^b_i$-sourced arrows; hence (by the induction hypothesis) $\id_{b_i}$ is both a forward-shift morphism from $\rF\id\id(b_i)$ to $b_i$ and a backward-shift morphism from $b_i$ to $\cB_\id(b_i)$.
\end{enumerate}
%
The other two clauses are proved by induction on the depth of $m$ and $n$. Let $o=o^m;o^n$ and for all $1\leq i\leq w$ let $v_i=v^n_{o^m(i)};v^m_i$. Note that $o$ is a function from $[1,w]$ to $[1,|e|]$ and all $v_i$ are arrows from $P^e_{o(i)}$ to $P^b_i$ as required.
%
\begin{enumerate}[resume]
\item For all $1\leq i\leq |b|$ and $1\leq j\leq |c|$, let $\cF^m_i$ and $\cF^n_j$ be the foward root shifters for $v^m_i$ and $v^n_j$, respectively, guaranteed by the fact that $m$ and $n$ are forward-shift morphisms, i.e., such that $m_i:\cF^m_i(c_{o^m(i)})\to b_i$ and $n_j:\cF^n_j(e_{o^n(j)})\to c_j$ are forward-shift morphisms.

\smallskip
For all $1\leq i\leq w$, we need to show the existence of a forward root shifter $\cF_i$ for $v_i$ such that $n_{o^m(i)};m_i$ is a morphism from $\cF_i(e_{o(i)})$ to $b_i$. Let $j=o^m(i)$ (hence $o^n(j)=o(i)$) and let $\cF_i=\cF^n_j;\cF^m_i$. By \pcite{ab-shifters compose}, $\cF_i$ is a forward root shifter for $v^n_j;v^m_i=v_i$. By definition, $n_j$ is a forward-shift morphism from $\cF^n_j(e_{o^n(j)})$ to $c_j$ and hence by \lcite{ab-root shifters preserve morphisms} (applying $\cF^m_i$) also a forward-shift morphism from $\cF_i(e_{o^n(j)})$ to $\cF^m_i(c_j)$. Again by definition, $m_i$ is a forward-shift morphism from $\cF^m_i(c_j)$ to $b_i$. By the induction hypothesis, therefore, $n_j;m_i$ is a forward-shift morphism from $\cF_i(e_{o(i)})$ to $b_i$.

\item Mutatis mutandis.
\qed
\end{enumerate}
\end{proof}
\end{fullorname}
%
\iffull
\begin{proposition}
For every $R$, $\AC R$ gives rise to categories $\ABC^\forw(R)$ with forward-shift morphisms and $\ABC^\back(R)$ with backward-shift morphisms.\todo{AR: Add to span-based, and add functiorality}
\end{proposition}
%
\fi
We now state the first main result, that ab-conditions endowed with either forward or backward morphisms give rise to a category with a functor to $\ABC^{\entails}$.

\begin{theorem}[categories of arrow-based conditions]\thlabel{ab-categories}
Let $\ABC^{\forw}$ be the category having ab-conditions as objects and forward-shift morphims as arrows, and let $\ABC^{\back}$ be the category having the same objects and backward-shift morphisms as arrows. They are well-defined, and there are identity-on-objects functors $\ABC^{\forw} \to \ABC^{\entails}$ and $\ABC^{\back} \to \ABC^{\entails}$.
\end{theorem}
%
\emph{Proof sketch.} Composition of morphisms and the fact that all identity morphisms have the correct nature for both categories was shown by \lcite{ab-morphisms compose}.
Moreover, identity morphisms and morphism composition satisfy the unit and associativity laws. The existence of the two identity-on-object functors follows from \pcite{ab-morphisms preserve models}.
\qed 

\medskip\noindent
Even though both forward- and backward-shift morphisms preserve models, they are incomparable: if two conditions are related by one kind of morphisms they are not necessarily related by the other kind. In other words, if we consider the existence of a morphism as providing evidence for entailment, then these two types of morphism are complementary, as the following example shows.
%
\begin{figure}[t]
\centering
\subcaptionbox
  {A forward-shift morphism without a backward-shift one
   \flabel{forward-no-backward}
  }
  [.45\textwidth]
  {\begin{tikzpicture}[on grid]
  \node[graph] (R1) {\oneedge x a z};
  \node[graph,below right=1.5 and 1.2 of R1] (10) {\oneedge{x}{a}{z}};
  \node[graph,below=1.5 of 10] (11) {\twoloop{x}{a}{z}{b}}; 

  \path (R1) edge[->] node[above right,inner sep=0] {$a^b_1$} (10)
        (10) edge[->] node[right] {$a^b_{11}$} (11);

  \node[graph,below left=1.5 and 1.5 of R1] (20) {\spangraph{y}{a}{x}{a}{z}};
  \node[graph,below=1.5 of 20] (21) {\mygraph{
    \node[n] (1) {$y$};
    \node[n] (2) [right=of 1] {$x$};
    \node[n] (3) [right=of 2] {$z$};
    \path (1) edge[<-] node[above] {\la} (2)
          (2) edge[->,bend left] node[near start,above] {\la} (3)
          (3) edge[->,bend left] node[near start,below] {\lb} (2);
    }};

  \path (R1) edge[->] node[above left,inner sep=0] {$a^c_1$} (20)
        (20) edge[->] node[left] {$a^c_{11}$} (21);
  
  \path[morphism]
        (20) edge[->] node[above] {$m_1$} (10)
        (11) edge[->] node[below] {$m_{11}$} (21);
	
\path[red,color=red]
  (10.south west) edge[->] node[above left,inner sep=0pt] {$\cF_{m_1}^s(a^c_{11})$} (21.20);
\end{tikzpicture}
}
  \qquad
\subcaptionbox
  {A backward-shift morphism without a forward-shift one
   \flabel{backward-no-forward}
  }
  [.45\textwidth]
  {\begin{tikzpicture}[on grid]
  \node[graph] (R) {\oneedge x a y};
  \node[graph,below left=1.5 and 1.5 of R] (20) {\mygraph{
    \node[n] (1) {$x$};
    \node[n] (2) [right=of 1] {$y$};
    \node[n] (3) [right=.5 of 2] {$z$};
    \path (1) edge[->] node[above] {\la} (2)
          (3) edge[loop right,->] node[right] {\lb} (3);
    }};
  \node[graph,below=1.5 of 20] (21) {\mygraph{
    \node[n] (1) {$x$};
    \node[n] (2) [right=of 1] {$y$};
    \path (1) edge[->,bend left] node[near start,above] {\la} (2)
          (2) edge[->,bend left] node[near start,below] {\lc} (1)
          (2) edge[loop right,->] node[right] {\lb} (2);
    }};

  \path (R) edge[->] node[above left,inner sep=0] {$a^c_1$} (20)
        (20) edge[->] node[left] {$a^c_{11}$} (21);
  
  \node[graph,below right=1.5 and 1.5 of R] (10) {\mygraph{
    \node[n] (1) {$x$};
    \node[n] (2) [right=of 1] {$y$};
    \path (1) edge[->] node[above] {\la} (2)
          (2) edge[loop right,->] node[right] {\lb} (2);
    }};
  \node[graph,below=1.5 of 10] (11) {\mygraph{
    \node[n] (1) {$x$};
    \node[n] (2) [right=of 1] {$y$};
    \node[n] (3) [left=of 1] {$z$};
    \path (1) edge[->] node[above] {\la} (2)
          (3) edge[->] node[above] {\lc} (1)
          (2) edge[loop right,->] node[right] {\lb} (2);
    }}; 

  \path (R) edge[->] node[above right,inner sep=0] {$a^b_1$} (10)
        (10) edge[->] node[right] {$a^b_{11}$} (11);

  \path[morphism]
        (20) edge[->] node[above] {$m_1$} (10)
        (11) edge[->] node[below] {$m_{11}$} (21);
		
\path[red,color=red]
  (20.south east) edge[->] node[above right,inner sep=0pt] {$\cB_{m_1}(a^b_{11})$} (11.north west);
\end{tikzpicture}
}
\caption{Forward-shift versus backward-shift morphisms}
\flabel{forward vs backward}
\end{figure}
%
\begin{example}[forward versus backward root shifters]\exlabel{forward vs backward}
\fcite{forward-no-backward} shows a forward-shift morphism where there is no backward-shift one. Both are evaluated in a context where we know $\la(x,z)$. In that context, $b$ is equivalent to the property $\neg \lb(z,x)$, and $c$ is equivalent to $\exists y\st \la(x,y) \wedge \neg \lb(z,x)$. Indeed, an intuitive explanation of $b\entails c$ is that $y$ can be instantiated with the value of $z$. The figure shows $\cF_{m_1}^s(a^c_{11})$ (for one of the sections $s$ of $m_1$), which acts as the identity on node and edges.

\fcite{backward-no-forward} shows a backward-shift morphism where there is no forward-shift one. Both are evaluated in a context where we know $\la(x,y)$. In that context, $b$ is equivalent to $\lb(y,y) \wedge \nexists z\st \lc(z,x)$ and $c$ is equivalent to $\exists z.\lb(z,z) \wedge \neg(y=z \wedge \lc(y,x))$.
Indeed, $b\entails c$ because $z$ can be instantiated with the value of $y$. The figure shows $\cB_{m_1}(a^c_{11})$.
\qed
\end{example}
%
As a final observation, given the fact that root shifters preserve morphisms (\lcite{ab-root shifters preserve morphisms}, those that are total are also functorial. ($\ABC^{\mathsf x}(X)$ demotes $\ABC^{\mathsf x}$ restricted to conditions rooted in $X$.

\begin{proposition}[source shifters yield functors]\plabel{ab-shifters are functors}
For any pair of objects $X,Y$ and any source shifter from $X$ to $Y$, the root shifter $\bar\cS$ is a functor from $\ABC^\forw(X)$ to $\ABC^\forw(Y)$ and from $\ABC^\back(X)$ to $\ABC^\back(Y)$.
\end{proposition}
